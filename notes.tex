\documentclass[reqno,12pt]{article}


\usepackage[utf8]{inputenc} % this is needed for umlauts
\usepackage[ngerman]{babel} % this is needed for umlauts
\usepackage[T1]{fontenc}    % this is needed for correct output of umlauts in pdf

\usepackage{fullpage}

%%%%%%%%%%%%%%%%%%%%%%%%%%%%%%%%%%%%%%%%%%%%%%%%%%%%%%%%%%%%%%%%%%%%%%%%%%%%%%%

\usepackage{amsmath, mathtools}
\usepackage{amssymb}
\usepackage[all]{xy}
\usepackage{hyperref}
\usepackage{comment}
\usepackage{color}
%%%%%%%%%%%%%%%%%%%%%%%%%%%%%%%%%%%%%%%%%%%%%%%%%%%%%%%%%%%%%%%%%%%%%%%%%%%%%%%

\numberwithin{equation}{section}
\setcounter{tocdepth}{1}
\setcounter{secnumdepth}{3}

%%%%%%%%%%%%%%%%%%%%%%%%%%%%%%%%%%%%%%%%%%%%%%%%%%%%%%%%%%%%%%%%%%%%%%%%%%%%%%%

%%%%%%%%%%%%%%%%%%%%%%%%%%%%%%%%%%%%%%%%%%%%%%%%%%%%%%%%%%%%%%%%%%%%%%%%%%%%%%
\newcommand{\bA}{\mathbb{A}}
\newcommand{\bD}{\mathbb{D}}
\newcommand{\bC}{\mathbb{C}}
\newcommand{\bF}{\mathbb{F}}
\newcommand{\bG}{\mathbb{G}}
\newcommand{\bP}{\mathbb{P}}
\newcommand{\bQ}{\mathbb{Q}}
\newcommand{\bR}{\mathbb{R}}
\newcommand{\bT}{\mathbb{T}}
\newcommand{\bZ}{\mathbb{Z}}
\newcommand{\bN}{\mathbb{N}}
%%%%%%%%%%%%%%%%%%%%%%%%%%%%%%%%%%%%%%%%%%%%%%%%%%%%%%%%%%%%%%%%%%%%%%%%%%%%%%%
\newcommand{\cA}{\mathcal{A}}
\newcommand{\cB}{\mathcal{B}}
\newcommand{\cC}{\mathcal{C}}
\newcommand{\cD}{\mathcal{D}}
\newcommand{\cE}{\mathcal{E}}
\newcommand{\cG}{\mathcal{G}}
\newcommand{\cF}{\mathcal{F}}
\newcommand{\cL}{\mathcal{L}}
\newcommand{\cN}{\mathcal{N}}
\newcommand{\cM}{\mathcal{M}}
\newcommand{\cO}{\mathcal{O}}
\newcommand{\cP}{\mathcal{P}}
\newcommand{\cT}{\mathcal{T}}
\newcommand{\cV}{\mathcal{V}}
\newcommand{\cU}{\mathcal{U}}
\newcommand{\cX}{\mathcal{X}}
%%%%%%%%%%%%%%%%%%%%%%%%%%%%%%%%%%%%%%%%%%%%%%%%%%%%%%%%%%%%%%%%%%%%%%%%%%%%%%%
\newcommand{\wt}{\widetilde}
\newcommand{\mc}{\mathcal}
\newcommand{\ol}{\overline}
\newcommand{\nd}{\noindent}
\newcommand{\iso}{\simeq}
%\renewcommand{\proof}{\nd \textbf{Beweis}.\,\,}
%%%%%%%%%%%%%%%%%%%%%%%%%%%%%%%%%%%%%%%%%%%%%%%%%%%%%%%%%%%%%%%%%%%%%%%%%%%%%%%
\newcommand{\Spec}{\text{Spec}\,}
\newcommand{\Id}{\text{Id}\,}
\newcommand{\id}{\text{id}}
\newcommand{\pt}{\text{pt}}
\newcommand{\Sm}{\text{Sm}}
\newcommand{\Stacks}{\underline{Stacks}}
\newcommand{\Set}{\underline{Set}}
\newcommand{\Sch}{\underline{Sch}}
\newcommand{\Cat}{\underline{Cat}}
\newcommand{\Grpd}{\underline{Grpd}}


\newcommand{\trdeg}{\text{trdeg}}
\newcommand{\Div}{\text{Div}}
\newcommand{\Pic}{\text{Pic}}
\newcommand{\Cl}{\text{Cl}}
\newcommand{\Alg}{\text{Alg}}
\newcommand{\Hom}{\text{Hom}}
\newcommand{\Rep}{\text{Rep}}
\newcommand{\Sym}{\text{Sym}}
\newcommand{\Coh}{\text{Coh}}
\newcommand{\PsCoh}{\text{PsCoh}}
\newcommand{\Perf}{\text{Perf}}
\newcommand{\LocFree}{\text{LocFree}}
%\renewcommand{\det}{\text{det}\,}

\renewcommand{\k}{\kappa}

\newcommand{\Ker}{\text{Ker }}
\renewcommand{\Im}{\text{Im }}


\newcommand{\slope}{\text{slope}}

\newcommand{\dt}{\overset{L}{\otimes}}
\newcommand{\D}{\Delta}

\newcommand{\edit}[1]{{\color{blue} #1}}
\newcommand{\fnote}[1]{\footnote{\color{red} #1}}
\newcommand{\red}[1]{\color{red} #1}
\newcommand{\blue}[1]{\color{blue} #1}

\usepackage{wasysym}

\usepackage{hyperref}



\usepackage{amsmath, amsfonts,epsfig, amssymb,amsbsy,eucal,mathrsfs,float, amsthm,mathtools, stmaryrd}

\setcounter{MaxMatrixCols}{20}


\theoremstyle{plain}

\newtheorem{thm}{Theorem}[section]
\newtheorem{lemma}[thm]{Lemma}
\newtheorem{proposition}[thm]{Satz}
\newtheorem{cor}[thm]{Korollar}
\newtheorem{conj}[thm]{Conjecture}


\theoremstyle{definition}

\newtheorem{definition}[thm]{Definition}
\newtheorem{example}[thm]{Beispiel}
\newtheorem{remark}[thm]{Bemerkung}
\newtheorem{notation}[thm]{Notation}




\newcommand{\blank}{{-}}

\newcommand{\End}{\operatorname{End}}

\renewcommand{\Ker}{\operatorname{Ker}}
\renewcommand{\Im}{\operatorname{Im}}
\newcommand{\Coker}{\operatorname{Coker}}
\newcommand{\Coim}{\operatorname{Coim}}

\mathchardef\mhyphen="2D
\newcommand{\Mod}{\text{Mod}}

\hypersetup{
%colorlinks=false,
colorlinks=true,
urlcolor=cyan
}

\newcommand{\fm}{\mathfrak{m}}
\newcommand{\fp}{\mathfrak{p}}
\newcommand{\fq}{\mathfrak{q}}
\newcommand{\fa}{\mathfrak{a}}
\newcommand{\fb}{\mathfrak{b}}



\begin{document}

\thispagestyle{empty}

\
\vspace{100pt}

\begin{center}
{\huge \bf Kommutative Algebra} \\
\vspace{10pt}
{\Large Maxim Smirnov} \\
\vspace{10pt}
{\large Universität Augsburg, Wintersemester 2023/2024 \\
für Bachelor und Lehramt \\
}
\vspace{10pt}
Draft \today
\end{center}



\newpage

{\footnotesize
\setcounter{tocdepth}{2}
\tableofcontents
}



\newpage

Hauptreferenz: \cite{AM}.


\newpage

\section{Ringe}

\subsection{Ringe und Homomorphismen}



\begin{definition}
Ein {\sf Ring} ist ein Tupel $(R, + , \cdot, 0, 1)$ bestehend aus einer Menge $R$, zweistelligen Verknüpfungen $+$ (Addition) und $\cdot$ (Multiplikation) und Elementen $0,1 \in R$ mit den folgenden Eigenschaften:
\begin{enumerate}
\item[{\bf R1}]   $(R, + , 0)$ ist eine abelsche Gruppe (d.h. die Addition ist assoziativ, die Null ist das neutrale Element diesbezüglich, und jedes $x \in R$ hat ein Inverses $-x$).
\item[{\bf R2}]   Die Multiplikation $\cdot \colon R \times R \to R$ ist assoziativ
\begin{align*}
  a \cdot  (b \cdot c) = (a \cdot b) \cdot c  \quad \forall a,b,c, \in R.
\end{align*}
\item[{\bf R3}] Das Element $1 \in R$, genannt {\sf die Eins}, ist ein neutrales Element bezüglich der Multiplikation
\begin{align*}
  x \cdot 1 = 1 \cdot x = x \quad \forall x \in R.
\end{align*}
Dieses ist eindeutig definiert.

\item[{\bf R4}]   Distributivität: für alle $a,b,c \in R$ gilt
\begin{align*}
 a \cdot (b+c) = a \cdot b + a \cdot c, \\
 (b+c) \cdot a = b \cdot a + c \cdot a.
\end{align*}

\item[{\bf R5}] Ein Ring heißt {\sf kommutativ}, wenn die Multiplikation kommutativ ist, d.h.
$$
x \cdot y = y \cdot x \quad \forall x,y \in R.
$$
\end{enumerate}

\end{definition}

\begin{remark}
\
\begin{enumerate}
  \item Wir werden fast ausschließlich  kommutative Ringe betrachten, deswegen haben wir die Kommutativität in die Definition mit integriert.
  \item Die obige Definition des Ringes schließt die Möglichkeit $1=0$ nicht aus. In diesem Fall folgt sofort, dass gilt $R=\{0\}$. Dieser Ring wird der {\sf Nullring} genannt.
\end{enumerate}
\end{remark}




\begin{example}
\
\begin{enumerate}
  \item $\bZ$.
  \item Jeder Körper ist ein Ring (z.B. $\bQ$, $\bR$, $\bC$).
  \item Polynomringe (später).
  \item Funktionen auf einer Menge $X$ mit Werten in $\bR$ mit der üblichen punktweisen Addition und Multiplikation.
\end{enumerate}
\end{example}

\begin{definition}
Seien $R$ und $S$ zwei Ringe. Eine Abbildung $ f \colon R \to S$ heißt {\sf Ringhomomorphismus}, wenn $f$ mit den vorhandenen Strukturen verträglich ist, d.h.
\begin{align*}
  & f(a+b)= f(a) + f(b) \\
  % & f(0) = 0 \\
  % & f(-a) = -f(a) \\
  & f(ab) = f(a)f(b) \\
  & f(1) = 1
\end{align*}
\end{definition}



\begin{example}
\
\begin{enumerate}
\item Für jeden Ring $R$ es gibt einen eindeutigen Ringhomomorphismus in den Nullring.
\item Für jeden Ring $R$ es gibt einen eindeutigen Ringhomomorphismus von $\bZ$ nach $R$.
\item Seien $A$ der Ring der $\bR$-wertigen Funktionen auf einer Menge $X$ und $P\in X$ ein Punkt. Dann definiert man den Auswertungshomomorphismus
\begin{align*}
&   \varphi \colon A \to \bR \\
&   f \mapsto  f(P).
\end{align*}
\end{enumerate}
\end{example}




\subsection{Polynomringe}

Sei $A$ ein Ring.

\medskip

1)  Ein Polynom in der Variablen $x$ mit Koeffizienten in $A$ ist ein Ausdruck der Form
\begin{align*}
  a_n x^n + a_{n-1} x^{n-1} + \dots + a_1 x + a_0,
\end{align*}
wobei $a_i \in A$. Die Addition und die Multiplikation definiert man auf die übliche Weise. Die Menge $A[x]$ aller Polynome in $x$ mit Koeffizienten in $A$ wird somit ein Ring.

Beachten Sie, dass die Polynome a priori keine Funktionen sind. Gegeben ein Polynom $P(x) \in A[x]$ kann man eine Funktion definieren
\begin{align*}
& \wt{P} \colon A \to A \\
&  \wt{P}(a):= P(a).
\end{align*}
Es kann aber passieren, dass zwei unterschiedliche Polynome dieselbe Funktion definieren.

\medskip

2) Ein Polynom in den Variablen $x_1, \dots, x_n$ mit Koeffizienten in $A$ ist eine endliche Summe der Form
\begin{align*}
  \sum_{i_1, \dots, i_n}  a_{i_1, \dots, i_n} x_1^{i_1} \dots x_n^{i_n}
\end{align*}
Mit der üblichen Addition und Multiplikation ist die Menge $A[x_1, \dots, x_n]$ aller Polynome ein Ring.



\subsection{Ringe der formalen Potenzreihen}

Eine formale Potenzreihe in $x$ über einem Ring $A$ ist ein Ausdruck der Form
\begin{align*}
\sum_{i=0}^{\infty} a_i x^i = a_0 + a_1x+a_2x^2 \dots ,
\end{align*}
wobei $a_i \in A$. Die Addition wird definiert als
\begin{align*}
  \Big(\sum_{i=0}^{\infty} a_i x^i \Big) + \Big(\sum_{i=0}^{\infty} b_i x^i\Big) = \sum_{i=0}^{\infty} (a_i+b_i) x^i.
\end{align*}
Die Multiplikation wird definiert als
\begin{align*}
  \Big(\sum_{i=0}^{\infty} a_i x^i \Big) \cdot \Big(\sum_{i=0}^{\infty} b_i x^i\Big) = a_0b_0 + (a_1b_0 + a_0b_1)x + (a_2b_0 + a_1b_1 + a_0b_2)x^2 + \dots.
\end{align*}
Mit diesen Verknüpfungen ist $A[[x]]$ ein kommutativer Ring.



\subsection{Ideale und Faktorringe}

\begin{definition}
Sei $R$ ein Ring und $S \subset R$ eine Untergruppe von $R$ bezüglich der Addition.

\begin{enumerate}
  \item $S$ heißt ein {\sf Unterring}, wenn $S \cdot S \subset S$ und $1 \in S$ (d.h. $S$ ist ein Ring bezüglich der induzierten Multiplikation).

  \item $S$ heißt ein {\sf Ideal}, wenn $S \cdot R \subset S$.
\end{enumerate}
\end{definition}

\begin{definition}
\
\begin{enumerate}
  \item Zu jeder Familie $(a_i)_{i \in I}$ von Elementen $a_i \in R$ hat man das von den $a_i$ erzeugte Ideal
  \begin{align*}
    ( a_i \mid i \in I ).
  \end{align*}
  Das ist das kleinste Ideal, welches alle $a_i$ enthält. Man sieht leicht, dass es aus allen Linearkombinationen
  \begin{align*}
    r_1a_{i_1} + \dots + r_na_{i_n} \, , \qquad r_j \in R \, , i_k \in I
  \end{align*}
  besteht.

  \item Ein Ideal $J$ heißt endlich erzeugt, wenn es endlich viele Elemente $a_1, \dots a_n$ gibt, sodass gilt
  \begin{align*}
    J = (a_1, \dots , a_n).
  \end{align*}

  \item Eine besondere Rolle spielen die Ideale, die nur einen Erzeuger haben, d.h.
  \begin{align*}
    J = (x)= Rx.
  \end{align*}
   Diese heißen {\sf Hauptideale}.
\end{enumerate}



\end{definition}

\begin{example}
\
\begin{enumerate}
  \item $\bZ \subset \bQ$ ist ein Unterring, aber kein Ideal.
  \item $2\bZ \subset \bZ$ ist kein Unterring, da $1 \notin 2 \bZ$, aber ein Ideal.
  \item Alle Ideal von $\bZ$ sind Hauptideale (Übungsblatt 1).
\end{enumerate}
\end{example}


\begin{lemma}
Sei $f \colon R \to S$ ein Ringhomomorphismus, dann ist das {\sf Bild} von $f$
\begin{align*}
  \Im f := \{y \in S \mid \exists x \in R \colon f(x)=y \}
\end{align*}
ein Unterring von $S$ und der {\sf Kern} von $f$
\begin{align*}
  \Ker f := \{x \in R \mid  f(x)=0 \}
\end{align*}
ist ein Ideal von $R$.
\end{lemma}

\begin{proof}
Gemacht in der Vorlesung.
\end{proof}

\begin{lemma}
Ein Ringhomomorphismus $f \colon R \to S$ ist genau dann injektiv, wenn gilt $\Ker f =0$.
\end{lemma}

\begin{proof}
Übungsblatt 1.
\end{proof}



Jetzt werden wir sehen, dass jedes Ideal als Kern eines Ringhomomorphismus entsteht.

\begin{definition}
Sei $R$ ein Ring und $I \subset R$ ein Ideal.

Erst betrachten wir $R$ als eine abelsche Gruppe bezüglich der Addition und $I \subset R$ als eine Untergruppe. Dann ist $I \subset R$ automatisch ein Normalteiler und man kann die Faktorgruppe $R/I$ bilden. Die Elemente von $R/I$ sind Nebenklassen $x+I$ mit $x \in R$ und die Addition wird als
\begin{align*}
  (x+I) + (y+I) = (x+y) + I
\end{align*}
definiert. Dadurch bekommen wir eine neue abelsche Gruppe $R/I$. Die Null in $R/I$ ist die Nebenklasse $0+I$.

Da $R$ eigentlich ein Ring ist und $I \subset R$ ein Ideal ist, können wir die Faktorgruppe $R/I$ mit einer Multiplikation ausstatten
\begin{align*}
& (a + I) \cdot (b + I) := ab + I \\
& 1_{R/I} := 1 + I.
\end{align*}
Die Wohldefiniertheit folgt daraus, dass $I$ ein Ideal ist.

Es gibt einen natürlichen Ringhomomorphismus
\begin{align*}
&  \pi \colon R \to R/I \\
& \quad  x \mapsto x + I,
\end{align*}
und man sieht leicht, dass $\pi$ surjektiv ist und es gilt
\begin{align*}
&  \Ker \pi = I.
\end{align*}
Der Ring $R/I$ wird {\sf Faktorring} genannt. Den Homomorphismus $\pi \colon R \to R/I$ nennt man {\sf kanonische Projektion}.
\end{definition}

\begin{proposition}[Homomorphiesatz]
Sei $R \to S$ ein Ringhomomorphismus. Dann gibt es einen natürlichen Isomorphismus
\begin{align*}
  R/\Ker f \overset{\simeq}{\to} \Im f
\end{align*}
\end{proposition}

\begin{proof}
Betrachten wir das kommutative Diagramm
\begin{align*}
  \xymatrix{
  R \ar[r]^f \ar[d]^{\pi}& S \\
  R/\Ker f \ar@{-->}[ur]_{\ol{f}}
  }
\end{align*}
Es gibt einen eindeutigen wohldefinierten Homomorphismus $\ol{f}$, da folgendes wegen der Kommutativität gelten muss: $\ol{f}(a+I) = f(a)$.

Nach Konstruktion ist $\ol{f}$ injektiv. Das Bild von $\ol{f}$ ist gleich dem Bild von $f$. Dadurch erhalten wir die Aussage.
\end{proof}


\begin{proposition}[Korrespondenzsatz für Ideale]
  Sei $R$ ein Ring und $I \subset R$ ein Ideal. Dann induziert die kanonische Projektion $\pi \colon R \to R/I$ eine Bijektion zwischen Idealen in $R/I$ und Idealen in $R$, die $I$ enthalten.
\end{proposition}

\begin{proof}
Die Bijektion ist gegeben durch
\begin{align*}
  \alpha \mapsto \pi^{-1} (\alpha). \hspace{50pt} (*)
\end{align*}
Die Injektivität von $(*)$ folgt sofort aus: $\pi(\pi^{-1} (\alpha)) = \alpha$. Die Surjektivität folgt aus der Tatsache, dass für ein Ideal $J \subset R$, das $I$ enthält, die Faktorgruppe $J/I$ ein Ideal in $R/I$ ist.
\end{proof}








\subsection{Nullteiler. Nilpotente. Einheiten.}


\begin{definition}
Ein Element $x \in R$ heißt {\sf Nullteiler}, wenn in $R$ ein $y \neq 0$ existiert, sodass $xy=0$. Ein Ring $R \neq 0$ ohne Nullteiler ($\neq 0$) heißt {\sf Integritätsbereich}.
\end{definition}
Z.B. sind $\bZ$ und $k[x_1, \dots, x_n]$ (hier ist $k$ ein Körper) Integritätsbereiche.


\begin{definition}
Ein Element $x \in R$ heißt {\sf nilpotent}, wenn es ein $n>0$ gibt, sodass gilt $x^n=0$.
\end{definition}

Jedes nilpotente Element ist ein Nullteiler (wenn $R \neq 0$). Die Umkehrung ist aber falsch (Übungsblatt 1).


\begin{definition}
Ein Element $x \in R$ heißt {\sf Einheit}, wenn es bezüglich der Multiplikation invertierbar ist, d.h. es existiert ein $y$  in $R$, sodass  $xy=1$. Das Element $y$ ist in diesem Fall eindeutig definiert und wird mit $x^{-1}$ bezeichnet.
\end{definition}

\begin{lemma}
  Eine formale Potenzreihe $\sum_{i=0}^{\infty} a_i x^i \in A[[x]]$ ist genau dann invertierbar, wenn $a_0 \in A$ invertierbar ist.
\end{lemma}
\begin{proof}
Übungsblatt 1.
\end{proof}

\begin{example}
\begin{align*}
  \frac{1}{1-x} = 1 + x + x^2 + \dots = \sum_{i=0}^{\infty} x^i
\end{align*}
\end{example}




\begin{definition}
Ein Ring $R \neq 0$ heißt {\sf Körper}, wenn jedes $x \in R \setminus \{0\}$ eine Einheit ist.
\end{definition}


\begin{proposition}
Sei $R$ ein Ring $\neq 0$. Die folgenden Aussagen sind äquivalent:
\begin{enumerate}
  \item $R$ ist ein Körper;
  \item die einzigen Ideale in $R$ sind $(0)$ und $(1)$;
  \item jeder Homomorphismus $\varphi \colon R \to S$ mit $S \neq 0$ ist injektiv.
\end{enumerate}
\end{proposition}

\begin{proof}
  $1) \Rightarrow 2)$ Sei $I \neq 0$ ein Ideal von $R$. Dann enthält $I$ ein Element $x \neq 0$. Da $x$ eine Einheit ist, gilt $I \supset (x) = R$. Somit gilt $I=R$.

\smallskip

  $2) \Rightarrow 3)$ Betrachten wir den Kern von $\varphi$. Da der Kern ein Ideal ist, haben wir $\Ker \varphi = (0)$ oder $\Ker \varphi = (1)$. Da die zweite Variante nur im Fall $S=0$ möglich ist, wird diese ausgeschlossen. Daraus folgt die Injektivität.

\smallskip

  $3) \Rightarrow 1)$ Sei $x \in R$ eine Nichteinheit. Dann gilt $(x) \neq (1)$ und somit ist $S=R/(x)$ kein Nullring. Dann ist die kanonische Projektion $R \to R/I$ injektiv. Daraus folgt $(x)=0$.
\end{proof}



\subsection{Primideale und maximale Ideale}

\begin{definition}
Sei $I \subset R$ ein echtes Ideal. Das Ideal $I$ heißt
\begin{enumerate}
  \item {\sf maximal} $\iff$ es existiert kein echtes Ideal $J$ mit $I \subset J$ und $I \neq J$.
  \item {\sf Primideal} $\iff$ wenn gilt $xy \in I$, dann gilt $x \in I$ oder $y \in I$.
\end{enumerate}
\end{definition}

\begin{proposition}\label{satz-characterisation-of-maxima/prime-ideals}
Sei $I \subset R$ ein Ideal.
\begin{enumerate}
  \item $I$ ist maximal $\iff$ $R/I$ ist ein Körper.
  \item $I$ ist ein Primideal $\iff$ $R/I$ ist ein Integritätsbereich.
\end{enumerate}
\end{proposition}

\begin{proof}
1)  Korrespondenzsatz für Ideale.

\smallskip

2) Folgt sofort aus der Definition: $[x][y]=0 \iff [x]=0 \text{ oder } [y]=0$.
\end{proof}

\begin{remark}
Nach diesem Satz sind alle maximalen Ideale auch Primideale. Die Umkehrung ist aber falsch (z.B. $(0) \subset \bZ$).
\end{remark}

\begin{proposition}\label{satz-preimages-of-prime-ideals}
Sei $f \colon R \to S$ ein Ringhomomorphismus und $I \subset S$ ein Primideal. Dann ist $f^{-1}(I) \subset R$ auch ein Primideal.
\end{proposition}

\begin{proof}
Erstens ist das Urbild eines Ideals ein Ideal, d.h. $f^{-1}(I)$ ist ein Ideal von $R$. Zweitens betrachten wir das folgende kommutative Diagramm (Homomorphiesatz anwenden):
\begin{align*}
  \xymatrix{
  R \ar[r]^f  \ar@/^2pc/[rr]^{\psi} \ar[d] & S \ar[r] & S/I \\
  R/\Ker \psi \ar@{-->}[urr]_{\ol{\psi}}
  }
\end{align*}
Laut Homomorphiesatz ist $\ol{\psi}$ injektiv. Da $\Ker \psi = f^{-1}(I)$, können wir $R/\Ker \psi = R/f^{-1}(I)$ als Unterring von $S/I$ betrachten. Laut Satz \ref{satz-characterisation-of-maxima/prime-ideals} ist $S/I$ ein Integritätsbereich und somit ist $R/f^{-1}(I)$ es auch. Fertig nach Satz \ref{satz-characterisation-of-maxima/prime-ideals}.
\end{proof}

\begin{remark}
Die Aussage des Satzes gilt nicht für maximale Ideale (z.B. $\bZ \subset \bQ$).
\end{remark}

Primideale spielen eine besondere Rolle in der kommutativen Algebra und in der algebraischen Geometrie. Das folgende Resultat garantiert, dass jeder Ring ($\neq 0$) mindestens ein Primideal hat.

\begin{thm}\label{theorem-maximal-ideals-exist}
Jeder Ring $R \neq 0$ hat ein maximales Ideal.
\end{thm}

\begin{proof}
Das Lemma von Zorn\footnote{Sei $P$ eine nichtleere partiell geordnete Menge, in der jede total geordnete Teilmenge (=eine Kette) eine obere Schranke hat. Dann enthält $P$ mindestens ein maximales Element.} auf das (nichtleere) partiell geordnete Menge von echten Idealen in $R$ anwenden. Gegeben eine aufsteigende Kette von solchen Idealen $\alpha_i$, deren obere Schranke ist gegeben durch $\cup_i \alpha_i$.
\end{proof}


\newpage
\begin{remark}
Für noethersche Ringe kann man das Theorem ohne Lemma von Zorn beweisen (später).
\end{remark}

\begin{cor}\label{cor-ideal-contained-in-maximal}
  Für jedes Ideal $I \subset R$ mit $I \neq R$ existiert ein maximales Ideal, in welchem $I$ enthalten ist.
\end{cor}

\begin{proof}
  Das Theorem auf den Ring $R/I$ anwenden und den Korrespondenzsatz für Ideale benutzen.
\end{proof}

\begin{cor}
  Jede Nichteinheit ist in einem maximalen Ideal enthalten.
\end{cor}

\begin{proof}
  Das Ideal $(x)$ betrachten und Korollar anwenden.
\end{proof}

\begin{definition}
  Ein Ring $R$ heißt {\sf lokal}, wenn es in $R$ genau ein maximales Ideal gibt.
\end{definition}

\begin{example}
  Jeder Körper ist lokal, $k[[x]]$ ist lokal, $k[x]/x^n$ ist lokal.
\end{example}

\begin{proposition}
\
  \begin{enumerate}
    \item Sei $R$ ein Ring und $\mathfrak{m} \neq (1)$ ein Ideal, sodass jedes $x \in R \setminus \mathfrak{m}$ eine Einheit ist. Dann ist $R$ ein lokaler Ring mit dem maximalen Ideal $\mathfrak{m}$.

    \item Sei $R$ ein Ring und $\mathfrak{m}$ ein maximales Ideal von $R$, sodass jedes Element von $1+\mathfrak{m}$ eine Einheit ist. Dann ist $R$ ein lokaler Ring.
  \end{enumerate}
\end{proposition}

\begin{proof}
  1) Elemente eines Ideals $\neq (1)$ sind immer Nichteinheiten. Somit müssen alle Ideale in $\mathfrak{m}$ enthalten sein.

  2) Betrachten wir ein $x \in R \setminus \mathfrak{m}$ und zeigen, dass es eine Einheit sein muss. Da $x \notin \mathfrak{m}$ und $\mathfrak{m}$ ein maximales Ideal ist, erzeugen (als Ideal) $\mathfrak{m}$ und $x$ den ganzen Ring $R$. Somit gilt $1 = xy + t$ für $y \in R$ und $t \in \mathfrak{m}$. Daraus folgt $xy \in 1 + \mathfrak{m}$ und ist somit eine Einheit. Somit ist auch $x$ eine Einheit. Jetzt Teil 1 anwenden.
\end{proof}



\begin{lemma}
  In einem Hauptidealbereich ist jedes Primideal entweder $(0)$ oder ein maximales Ideal.
\end{lemma}

\begin{proof}
  Da $R$ ein Integritätsbereich ist, ist $(0)$ ein Primideal (Satz \ref{satz-characterisation-of-maxima/prime-ideals}).

  Sei jetzt $\mathfrak{p}=(x)$ ein Primideal $\neq 0$ und $(y) \supset (x)$ ein echtes Ideal, welches $(x)$ enthält. Dann haben wir: $x = ya \Rightarrow a \in (x) \Rightarrow  a = xb \, \Rightarrow \, x = yxb \, \Rightarrow \, x (1-yb) = 0 \, \Rightarrow \, yb = 1 \, \Rightarrow \, (y)=R$.
\end{proof}

\begin{example}
  $\bZ$ und $k[x]$ sind Hauptidealbereiche.
\end{example}


\subsection{Nilradikal}

\begin{proposition}
  Die Menge $\mathfrak{N}$ aller nilpotenten Elemente in einem Ring $R$ ist ein Ideal. Der Faktorring $R/\mathfrak{N}$ hat keine Nilpotente außer $0$.
\end{proposition}

\begin{proof}
  Sei $x$ ein Nilpotent. Dann ist das Element $ax$ auch nilpotent für jedes $a \in R$. Seien $x,y$ zwei Nilpotente: $x^m = 0$ und $y^n=0$. Dann ist auch $(x+y)^{m+n} = 0$. Somit ist $\mathfrak{N}$ ein Ideal.

  Sei $\ol{x}=x+\mathfrak{N} \in R/\mathfrak{N}$ ein Nilpotent, d.h. $\ol{x}^n = 0$ für ein $n$. Somit gilt $x^n \in \mathfrak{N}$. Dann ist $x$ auch ein Nilpotent in $R$, und somit in $\mathfrak{N}$ enthalten ist. Daraus folgt $\ol{x}=0$.
\end{proof}


\newpage
\begin{proposition}\label{satz-nilradical}
  Sei $R$ ein Ring $\neq 0$. Es gilt
  \begin{align*}
    \mathfrak{N} = \bigcap_{\mathfrak{p} \in \text{Primideale von R}}   \mathfrak{p}.
  \end{align*}
\end{proposition}



\begin{proof}
Sei $f \in R$ ein Nilpotent und $\mathfrak{p}$ ein Primideal. Wir haben: $f^n = 0\in \mathfrak{p} \Rightarrow f \in \mathfrak{p}$. Somit ist $f$ in allen Primidealen enthalten.

Sei jetzt $f \in R$ kein Nilpotent. Wir möchten zeigen, dass es ein Primideal gibt, in welchem $f$ nicht enthalten ist. Sei $\Sigma$ die Menge aller Ideale $\alpha$ mit der Eigenschaft
$$
n > 0 \Rightarrow f^n \notin \alpha.
$$
Da $(0) \in \Sigma$ gilt, ist diese Menge nicht leer. Wie in Theorem \ref{theorem-maximal-ideals-exist} können wir hier auch das Lemma von Zorn anwenden und ein maximales Element von $\Sigma$ erhalten, das wir mit $\mathfrak{p}$ bezeichnen. Wir zeigen jetzt, dass $\mathfrak{p}$ ein Primideal ist. Seien $x,y \notin \mathfrak{p}$. Dann sind die Ideale $(x) + \mathfrak{p}$ und $(y) + \mathfrak{p}$ echt größer als $\mathfrak{p}$, und sind somit nicht in $\Sigma$ enthalten. Also gilt
$$
f^m \in (x) + \mathfrak{p} \quad \text{und} \quad f^n \in (y) + \mathfrak{p}
$$
für gewisse $m,n > 0$. Daraus folgt $f^{m+n} \in (xy) + \mathfrak{p}$. Somit ist $(xy) + \mathfrak{p}$ nicht in $\Sigma$ und~$xy \notin \mathfrak{p}$.
\end{proof}



\subsection{Operationen mit Ringen und Idealen}

\begin{definition}
1) Seien $I,J$ zwei Ideale. Die {\sf Summe} von $I$ und $J$ ist ein Ideal von $R$, definiert als
\begin{align*}
  I+J = \{ x+y \mid x \in I, y \in J \}.
\end{align*}
Das Ideal $I+J$ ist das kleinste Ideal, welches $I$ und $J$ enthält. Analog kann man die Summe einer beliebigen Familie $\{I_s\}_{s \in S}$ von Idealen definieren.

\smallskip

2) Seien $I,J$ zwei Ideale. Man definiert das {\sf Produkt} als das Ideal erzeugt von den Produkten $xy$ mit $x \in I$ und $y \in J$. Es gilt
\begin{align*}
  IJ = \big \{ \sum_{i=1}^n x_iy_i \mid x_i \in I, y_i \in J, n\in \mathbb{N} \big \}.
\end{align*}
Analog kann man das Produkt einer endlichen Familie $\{I_1, \dots, I_n\}$ von Idealen definieren. Insbesondere können wir über Potenzen $I^n = I \dots I$ sprechen.

\smallskip

3) Der {\sf Schnitt} $\cap_{s \in S} I_s$ einer beliebigen Familie $\{I_s\}_{s \in S}$ von Idealen ist ein Ideal.
\end{definition}


\begin{example}
  Sei $R=\bZ$. Dann gilt
  \begin{align*}
    & (a) + (b) = gcd (a,b) \\
    & (a) \cap (b) = lcm (a,b) \\
    & (a)(b) = (ab)
  \end{align*}
\end{example}


\begin{definition}
  Sei $I \subset R$ ein Ideal. Das Ideal
  \begin{align*}
    r(I) = \{x \in R \mid \exists n : x^n \in I \}
  \end{align*}
  nennt man {\sf Radikal} von $I$.
\end{definition}

\begin{example}
Sei $R=\bZ$ und $I = (18)$. Dann gilt $r(I) = (6)$.
\end{example}

\begin{proposition}
Das Radikal von $I$ ist der Schnitt aller Primideale, die $I$ enthalten.
\end{proposition}

\begin{proof}
  Satz \ref{satz-nilradical} auf $R/I$ anwenden (Übungsblatt 2).
\end{proof}

\begin{definition}
  Seien $R$ und $S$ Ringe. Das {\sf direkte Produkt} ist definiert als
  $$
  R \times S = \{ (x,y) \mid x \in R, y \in S \}
  $$
  mit der komponentenweisen Addition und Multiplikation. Die Null ist $(0,0)$ und die Eins ist~$(1,1)$.

  Das direkte Produkt ist mit den  kanonischen Projektionen
  \begin{align*}
  & p_1 \colon R \times S \to R \\
  & p_1(r,s)=r \\
  & p_2 \colon R \times S \to S \\
  & p_2(r,s)=s
  \end{align*}
  ausgestattet, und besitzt folgende {\sf universelle Eigenschaft}: Für jeden Ring $T$ und beliebige Ringhomomorphismen $f \colon T \to R$ und $g \colon T \to S$ gibt es einen eindeutigen Ringhomomorphismus $\psi \colon T \to R\times S$, sodass das folgende Diagramm kommutiert
  \begin{align*}
    \xymatrix{
     T \ar@{-->}[rr]^{\exists ! \psi } \ar[dr]_f \ar[drrr]_g &&   R \times S \ar[dl] \ar[dr]\\
     & R & & S
    }
  \end{align*}
Man kann zeigen, dass diese universelle Eigenschaft das direkte Produkt eindeutig definiert (Übungsblatt 3).

Analog kann man das Produkt einer beliebigen (auch unendlichen) Familie von Ringen definieren.
\end{definition}

\begin{example}
Es gilt $\bZ / (6) = \bZ/(2) \times \bZ/(3)$.
\end{example}

Allgemeiner gilt folgendes Lemma.

\begin{lemma}
  Sei $R$ ein Ring, $I_1, \dots, I_n$ Ideale und $\psi$ der kanonische Homomorphismus
  $$
  R \to R/I_1 \times \dots \times R/I_n.
  $$
  Dann gilt:
  \begin{enumerate}
    \item $I_a$ und $I_b$ sind koprim für $a \neq b$ $\Rightarrow$ $\prod_s I_s =  \cap_s I_s$
    \item $\psi$ ist surjektiv $\iff$ $I_a$ und $I_b$ sind koprim für $a \neq b$
    \item $\psi$ ist injektiv $\iff$ $\cap_s I_s = (0)$
  \end{enumerate}
\end{lemma}
\begin{proof}
  Übungsblatt 3.
\end{proof}



\subsection{Spektrum eines Ringes}\label{subsection-Spektrum}

\begin{definition}
Sei $R$ ein Ring. Die Menge aller Primideale wird {\sf Spektrum} von $R$ genannt und mit $\Spec R$ bezeichnet.
\end{definition}
Vorerst ist $\Spec R$ nur eine Menge. Als Nächstes definieren wir eine Topologie darauf.

\begin{definition}
  Eine {\sf Topologie} auf einer Menge $X$ ist ein Mengensystem $T$ bestehend aus Teilmengen von $X$, den sogenannten {\sf offene Teilmengen}, die die folgenden Eigenschaften erfüllen:
  \begin{enumerate}
    \item $\emptyset$ und $X$ sind offen;
    \item die Vereinigung beliebig vieler offener Teilmengen ist offen;
    \item der Schnitt endlich vieler offener Teilmengen ist offen.
  \end{enumerate}

  Eine Teilmenge $A \subset X$ heißt {\sf abgeschlossen}, wenn das Komplement $X \setminus A$ offen ist.

  \medskip

  Ein {\sf topologischer Raum} ist eine Menge $X$ ausgestattet mit einer Topologie $T$.
\end{definition}

%\begin{example}
%Keines gegeben. Siehe Beispiel \ref{example-spectra-of-rings}.
%\end{example}

\begin{remark}
  Äquivalent kann man abgeschlossene Teilmengen benutzen, um eine Topologie zu definieren. Für diese soll gelten:
  \begin{enumerate}
    \item $\emptyset$ und $X$ sind abgeschlossen;
    \item die Vereinigung endlich vieler abgeschlossener Teilmengen ist abgeschlossen;
    \item der Schnitt beliebig vieler abgeschlossener Teilmengen ist abgeschlossen.
  \end{enumerate}


\end{remark}


\noindent Jetzt sind wir bereit um eine Topologie auf $\Spec R$ zu definieren.

\begin{definition}
  Sei $R$ ein Ring und $E \subset R$ eine Teilmenge. Die Menge aller Primideale von $R$, die $E$ enthalten, bezeichnen wir mit $V(E)$. Diese wird {\sf Verschwindungsmenge} von $E$ genannt. Es ist leicht zu sehen, dass gilt $V(E) = V (\langle E \rangle)$. Somit genügt es $V(I)$, wobei $I \subset R$ ein Ideal ist, anzuschauen.
\end{definition}

\begin{lemma}\label{lemma-V(I)-define-topology}
Es gelten folgende Eigenschaften:
\begin{align*}
  & V(0) = \Spec R \quad \text{und}\quad V(R) = \emptyset \\
  & \cap_s V(I_i)  = V (\sum_s I_s) \\
  & V(I) \cup V(J) = V(IJ) \\
  & V (I) = V ( r(I) ) \\
\end{align*}
\end{lemma}

\begin{proof}
Die ersten zwei Punkte wurden in der Vorlesung bewiesen. Die letzten
beiden werden auf Übungsblatt 4 gestellt.
\end{proof}

\begin{definition}
Nach Lemma \ref{lemma-V(I)-define-topology} erfüllen die Mengen $V(I)$ die Eigenschaften der abgeschlossen Teilmengen einer Topologie auf $\Spec R$. Diese Topologie wird \href{https://en.wikipedia.org/wiki/Zariski_topology}{\sf Zariski-Topologie} genannt.
\end{definition}

\begin{example}
\label{example-spectra-of-rings}
\
  \begin{enumerate}
    \item Sei $R=K$ ein Körper. Das Nullideal ist das einzige Primideal. Somit hat die Menge $\Spec K$ nur einen Punkt. Die Topologie ist eindeutig.

    \item Sei $R=\bZ$. Die Primideale sind der Form $(p)$ mit $p$ einer Primzahl oder das Nullideal~$(0)$. Die abgeschlossenen Teilmengen sind der Form $V(n) = \{\text{Primteiler von $n$} \}$, die ganze Menge $\Spec \, \bZ$ und die leere Teilmenge $\emptyset$.

    Eine amüsante Eigenschaft dieser Topologie ist Folgendes: Die Punkte der Form $(p)$ mit einer Primzahl sind abgeschlossen. Im Gegensatz dazu ist der Punkt $(0)$ nicht abgeschlossen: Die kleinste abgeschlossene Teilmenge von $\Spec \, \bZ$, die $(0)$ enthält ist $\Spec \, \bZ$ selbst, d.h. der Punkt $(0)$ ist dicht in $\Spec \, \bZ$. Solche Punkte nennt man \href{https://en.wikipedia.org/wiki/Generic_point}{\sf generische Punkte}.

    \item Sei $R=\bC[x]$. Die Primideale sind von der Form $(x-a)$ mit $a \in \bC$ oder das Nullideal $(0)$. Die abgeschlossenen Teilmengen sind von der Form $V(f(x)) = \{\text{Nullstellen von $f(x)$} \}$. Auch in diesem Fall ist der Punkt $(0)$ ein generischer Punkt.
  \end{enumerate}
\end{example}

\medskip
\noindent \textbf{Motivation:} In der modernen algebraischen Geometrie ordnet man einem Ring $R$ ein geometrisches Objekt zu: Sein Spektrum $\Spec R$ ausgestattet mit der Zariski-Topologie und einer Garbe von Funktionen darauf (kommt später). Solche geometrischen Objekte nennt man affine Schemata. Ein allgemeines Schema wird dann als eine Verklebung von affinen Schemata definiert.



\begin{definition}\label{definition-induced-map-on-spectra}
Sei $f \colon R \to S$ ein Ringhomomorphismus. Dieser induziert eine natürliche Abbildung von Spektren
\begin{align}\label{eq.-induced-map-on-spectra}
  & \wt{f} \colon \Spec S \to \Spec R \\ \notag
  &  \qquad \mathfrak{p} \mapsto f^{-1}(\mathfrak{p})
\end{align}
(wohldefiniert nach Satz \ref{satz-preimages-of-prime-ideals}).
\end{definition}


\begin{definition}
  Eine Abbildung $\psi \colon X \to Y$ von topologischen Räumen heißt {\sf stetig}, wenn für jedes offenes $U \subset Y$ das Urbild $\psi^{-1}(U) \subset X$ auch offen ist.
\end{definition}


\begin{lemma}
Die Abbildung \eqref{eq.-induced-map-on-spectra} ist stetig.
\end{lemma}



\begin{proof}
Sei $I \subset R$ ein Ideal. Es gilt (Übungsblatt 4)
\begin{align*}
\wt{f}^{-1}\big( V(I) \big) = V(f(I)).
\end{align*}
Somit sind die Urbilder von abgeschlossenen Teilmengen abgeschlossen. Dann sind auch die Urbilder von offen Teilmengen offen.
\end{proof}












\section{Moduln}

\subsection{Moduln und Homomorphismen}

\begin{definition}
Sei $R$ ein Ring. Ein $R$-Modul ist eine abelsche Gruppe $M$ (additiv geschrieben) ausgestattet mit einer \glqq Wirkung\grqq\, von $R$
\begin{align*}
  & \mu \colon R \times M \to M \\
  & \quad  (a,m) \mapsto \mu(a,m) = a \cdot m
\end{align*}
die die folgenden Eigenschaften haben soll:
\begin{align*}
  & (a+b) \cdot x = a \cdot x + b \cdot x \\
  & a \cdot (x+y) = a \cdot x + a \cdot y \\
  & (a \cdot b) \cdot x = a \cdot (b \cdot x) \\
  & 1 \cdot x = x
\end{align*}
für alle $a,b \in R$ und $x,y \in M$.
\end{definition}


Der Begriff eines Moduls verallgemeinert mehrere bekannte Begriffe.

\begin{example}
\

\noindent Kommutative Beispiele:
  \begin{enumerate}
    \item Moduln über einem Körper $K$ sind $K$-Vektorräume.
    \item Moduln über $\bZ$ sind abelsche Gruppen.
    \item Jeder Ring $R$ ist ein $R$-Modul. Jedes Ideal $I \subset R$ ist ein $R$-Modul.
    \item Für jeden Ring $R$ ist das direkte Produkt $R^n$ auch ein $R$-Modul.
    \item Sei $f \colon R \to S$ ein Ringhomomorphismus. Dann können wir $S$ mit einer $R$-Modulstruktur ausstatten: $r \cdot m := f(r)m$ für $r \in R$ und $m \in S$.
    \item Sei $K$ ein Körper und $R=K[x]$. Ein Modul über $R$ ist ein $K$-Vektorraum $V$ mit einer $K$-linearen Abbildung $V \to V$.
  \end{enumerate}
\end{example}

\noindent {\bf Geometrische Motivation:} Sei $X$ ein Raum (topologischer Raum, glatte Mannigfaltigkeit, komplexe Mannigfaltigkeit, algebraische Varietät etc), $\cO(X)$ der Ring von globalen Funktionen auf $X$, und $E \to X$ ein Vektorbündel. Dann ist die Menge $E(X)$ der globalen Schnitte von $E$ ein Modul über $\cO(X)$.

\newpage

\begin{definition}
\
\begin{enumerate}
  \item Eine Abbildung $f \colon M \to N$ von $R$-Moduln heißt ein {\sf Homomorphismus von $R$-Moduln} (oder $R$-linear), wenn gilt
  \begin{align*}
    & f(x+y) = f(x) + f(y) \\
    & f(a \cdot x) = a \cdot f(x)
  \end{align*}
  für alle $x,y \in M$ und $a \in R$.

  \item Die Menge aller $R$-linearen Homomorphismen $f \colon M \to N$ bezeichnen wir mit
  $$
  \Hom_R(M,N).
  $$
  Das ist auch ein $R$-Modul (mit der punktweisen Addition und Wirkung von $R$).

  \item Die {\sf Endomorphismen}
  $$
  \End_R(M):= \Hom_R(M,M)
  $$
  bilden einen (nichtkommutativen!) Ring.

  \item Wie früher: ein Mono(Epi-, Iso-)morphismus ist ein injektiver (surjektiver, bijektiver) Homomorphismus.
\end{enumerate}

\end{definition}


\begin{example}
\
\begin{enumerate}
  \item Sei $R$ ein Ring, $M$ ein $R$-Modul und $a \in R$ ein Element. Dann ist die Abbildung $m \mapsto am$ ein Homomorphismus.

  \item Ein $f \in \Hom_R(R , M)$ ist eindeutig durch $f(1)$ definiert. Dadurch erhalten wir einen Isomorphismus von $R$-Moduln
  \begin{align*}
    \Hom_R(R , M) &\to  M \\
      \qquad f \qquad &\mapsto  f(1)
  \end{align*}

  \item Homomorphismen $R^n \to R^m$ sind durch $n \times m$-Matrizen gegeben.
  \item Sei $R$ ein Integritätsring und $I \subset R$ ein Ideal. Dann gilt $\Hom_R(R/I, R) = 0$.
\end{enumerate}
\end{example}







\newpage
\subsection{Untermoduln und Faktormoduln}

\begin{definition}
\
\begin{enumerate}
  \item Sei $M$ ein $R$-Modul. Eine Untergruppe $N \subset M$ heißt {\sf Untermodul}, wenn diese bezüglich der Multiplikation mit $R$ abgeschlossen ist.

  \item Sei $N \subset M$ ein Untermodul. Die Faktorgruppe $M/N$ hat eine natürliche $R$-Modul-Struktur: $a(m + N) = am + N$. Dieser Modul wird {\sf Faktormodul} genannt. Die natürliche Abbildung $M \to M/N$ ist ein Homomorphismus von $R$-Moduln.
\end{enumerate}
\end{definition}

\begin{definition}
Sei $f \colon M \to N$ ein Homomorphismus von $R$-Moduln, dann sind der {\sf Kern} und  das {\sf Bild}
\begin{align*}
&  \Ker f := \{x \in M \mid  f(x)=0 \} \\
&  \Im f := \{y \in N \mid \exists x \in M \colon f(x)=y \}
\end{align*}
Untermoduln von $M$ bzw. $N$.
\end{definition}


\begin{proposition}[Homomorphiesatz]
Sei $f \colon M \to N$ ein Homomorphismus von $R$-Moduln. Dann gibt es einen natürlichen Isomorphismus
\begin{align*}
  M/\Ker f \overset{\simeq}{\to} \Im f
\end{align*}
\end{proposition}

\begin{proof}
Übung.
\end{proof}


\begin{proposition}[Korrespondenzsatz]
  Sei $M$ ein $R$-Modul und $N \subset M$ ein Untermodul. Dann induziert die kanonische Projektion $\pi \colon M \to M/N$ eine Bijektion zwischen Untermoduln von $M/N$ und Untermoduln von $M$, die $N$ enthalten.
\end{proposition}

\begin{proof}
Übung.
\end{proof}


\begin{definition}
Sei $M$ ein $R$-Modul und $(M_i)_{i \in I}$ eine Familie von Untermoduln von $M$.
\begin{enumerate}
  \item Die {\sf Summe} $\sum_i M_i$ ist die Menge aller endlichen Summen $\sum_i x_i$ mit $x_i \in M_i$. Diese ist der kleinste Untermodul von $M$, welcher alle $M_i$ enthält.

  \item Der {\sf Schnitt} $\bigcap_{i \in I} M_i$ ist auch ein Untermodul von $M$.
\end{enumerate}

\end{definition}

\begin{proposition}
\
\begin{enumerate}
  \item Seien $M \supset M' \supset M''$ Moduln über $R$. Dann gilt
  \begin{align*}
  & (M/M'')/(M'/M'') \iso M/M'
  \end{align*}

  \item Seien $M_{1,2} \subset M$ Untermoduln. Dann gilt
  \begin{align*}
   \left( M_1 + M_2 \right) /M_1 \iso M_2/(M_1 \cap M_2).
  \end{align*}
\end{enumerate}

\end{proposition}

\begin{proof}
Übungsblatt 5.
\end{proof}







\subsection{Direkte Summe und direktes Produkt}

\begin{definition}
Sei $(M_i)_{i \in I}$ eine Familie von $R$-Moduln.
\begin{enumerate}
  \item Die {\sf direkte Summe} $\bigoplus_{i \in I} M_i$ wird definiert als
  \begin{align*}
  \bigoplus_{i \in I} M_i = \big \{ (m_i)_{i \in I} \mid m_i \in M_i, \text{ endliche viele $m_i \neq 0$} \big \}.
  \end{align*}

  \item Das {\sf direkte Produkt} $\prod_{i \in I} M_i$ wird definiert als
  \begin{align*}
  \prod_{i \in I} M_i = \big \{ (m_i)_{i \in I} \mid m_i \in M_i \big \}.
  \end{align*}
\end{enumerate}
Die beiden sind wieder $R$-Moduln mit den komponentenweisen Operationen. Ähnlich zu Ringen kann man die beiden auch durch universelle Eigenschaften definieren (später).
\end{definition}



\subsection{Freie Moduln. Endlich Erzeugte Moduln. Torsion.}



\begin{lemma}
  Sei $R \neq 0$ ein Ring und $R^n \iso R^m$ als $R$-Moduln. Dann gilt $m=n$.
\end{lemma}

\begin{proof}
  Seien $\mathfrak{m} \subset R$ ein maximales Ideal, $M=R^m$ und $N=R^n$. Ein Isomorphismus $M \iso N$ induziert einen Isomorphismus von Untermoduln\footnote{Für einen $R$-Modul $M$ und ein Ideal $\alpha \subset R$ definiert man den Untermodul $$\alpha M := \{ \sum_i a_im_i \mid  a_i \in \alpha, m_i \in M\}.$$} $\mathfrak{m} M \iso \mathfrak{m} N$. Somit erhalten wir einen Isomorphismus $M/ \mathfrak{m} M \iso N/ \mathfrak{m} N$. Also es gilt $(R/\mathfrak{m})^m \iso (R/\mathfrak{m})^n$ als $R$-Moduln. Daraus folgt, dass $(R / \mathfrak{m})^m \iso (R/ \mathfrak{m})^n$ auch als $R / \mathfrak{m}$-Vektorräume isomorph sind und wir erhalten $m=n$.
\end{proof}



\begin{definition}
Sei $M$ ein $R$-Modul.
\begin{enumerate}
  \item $M$ heißt {\sf frei} $\iff$  $\exists$ ein Isomorphismus $M \iso \bigoplus_{i \in I} R$.

  \item $M$ heißt {\sf frei vom Rang $n$} $\iff$ $card(I)=n$ ist endlich.\footnote{Wohldefiniert nach dem Lemma.}
\end{enumerate}
\end{definition}

\begin{example}
Eine Familie von Elementen $\{e_1, \dots, e_n \}$ von $M$ heißt {\sf Basis}, wenn jedes $m \in M$ eindeutig als Linearkombination $\sum_i a_i m_i$ mit $a_i \in R$ geschrieben werden kann.

\medskip

\noindent $M=R^n$: Man sieht leicht, dass die Elemente
\begin{align*}
& e_1 = (1,0, \dots, 0) \\
& e_2 = (0,1, \dots, 0) \\
& \dots \\
& e_n = (0, \dots, 0,1)
\end{align*}
von $R^n$ eine Basis bilden. Diese Basis werden wir {\sf Standardbasis} von $R^n$ nennen.
\end{example}


\begin{lemma}
Ein $R$-Modul $M$ ist frei vom Rang $n$ $\iff$ es gibt eine Basis von $M$ mit $n$ Elementen.
\end{lemma}

\begin{proof}
Übung.
\end{proof}

\begin{lemma}
  Sei $M$ ein $R$-Modul und $f \colon M \to R^n$ ein surjektiver Homomorphismus. Dann existiert eine Spaltung\footnote{D.h. $f \circ g = \id_{R^n}$.} $g \colon R^n \to M$ und ein Isomorphismus
  \begin{align*}
  h \colon M \to R^n \oplus \Ker f.
  \end{align*}
\end{lemma}

\begin{proof}
  Sei $R^n = R e_1 \oplus \dots \oplus R e_n$. Wähle $m_i \in M$ mit $f(m_i) = e_i$ und definiere $g \colon R^n \to M$ als
  \begin{align*}
  g(c_1e_1 + \dots + c_n e_n) = c_1 m_1 + \dots + c_n m_n.
  \end{align*}
  Es gilt  $f \circ g = \id_{R^n}$. Nun definiere $h \colon M \to R^n \oplus \Ker f$ als
  \begin{align*}
  h(m) = (f(m), g(f(m)) - m).
  \end{align*}
  Das ist ein Isomorphismus (Übungsblatt 5).
\end{proof}

\begin{remark}
  Nach Homomorphiesatz induziert jeder surjektive Homomorphismus $M \to N$ einen Isomorphismus $M/\Ker f \iso N$. Es ist aber sehr selten, dass es eine Zerlegung $M \iso N \oplus \Ker f$ gibt. Z.B. geht es für die kanonische Projektion $\bZ \to \bZ/2\bZ$ nicht (Übungsblatt~5).
\end{remark}





\begin{definition}
  Ein $R$-Modul $M$ heißt {\sf endlich erzeugt}, wenn es einen surjektiven Homomorphismus $R^n \to M$ gibt. Äquivalent: ein $R$-Modul $M$ ist endlich erzeugt, wenn es endlich viele Elemente $m_1 , \dots, m_n \in M$ gibt, sodass jedes $m \in M$ als $R$-Linearkombination von $m_i$'s geschrieben werden kann. Solche Elemente werden {\sf Erzeuger} von $M$ genannt (die Zahl $n$ ist nicht festgelegt).
\end{definition}

\begin{definition}
Sei $M$ ein $R$-Modul. Ein Element $m \in M$ heißt {\sf Torsionselement}, wenn es ein $a \neq 0$ in $R$ gibt, sodass gilt $am=0$. Ein Modul ohne Torsionselemente $\neq 0$ heißt {\sf torsionsfrei}.
\end{definition}

\begin{lemma}
Sei $R$ ein Integritätsbereich und $M$ ein $R$-Modul.
\begin{enumerate}
  \item Die Menge $M_{tors} \subset M$ aller Torsionselemente ist ein Untermodul von $M$.
  \item Der Faktormodul $M/M_{tors}$ ist torsionsfrei.
\end{enumerate}
\end{lemma}




\begin{lemma}
Sei $M$ ein endlich erzeugter torsionsfreier Modul über einem Integritätsbereich $R$. Dann existiert eine Einbettung\footnote{Einbettung = injektiver Homomorphismus.} $M \to R^d$.
\end{lemma}

\begin{proof}
Sei $K$ der Quotientenkörper von $R$ und $x_1, \dots, x_n$ Erzeuger von $M$.

\begin{enumerate}
  \item {\it Es gibt höchstens $n$ linear unabhängige Elemente in $M$:} Sei $f \colon R^n \to M$ der Homomorphismus definiert durch $f(e_i) = x_i$. Seien $y_1, \dots ,y_k$ linear unabhängig, dann ist der Untermodul aufgespannt von den $y_i$'s isomorph zu $R^k$. Man kann schreiben $y_i = \sum_{j} a_{ij} x_j$ und dann zu $R^n$ liften: $v_i := \sum_{j} a_{ij} e_j$. Es gilt $f(v_i) = y_i$. Da $y_1, \dots ,y_k$ linear unabhängig in $M$ sind, sind $v_1, \dots ,v_k$ linear unabhängig in $R^n$. Somit sind diese auch in $K^n$ linear unabhängig über $K$.

  \item Sei $t_1, \dots , t_d$ eine maximale (größtmögliche) Familie von linear unabhängigen Elementen. Dann gilt
  $$
  \sum_i R t_i \iso R^d.
  $$
  Für jedes $x \in M$ sind die Elemente $\{x, t_1, \dots, t_d\}$ linear abhängig. Somit gilt
  $$
  ax = \sum_i a_i t_i
  $$
  mit $a \neq 0$. D.h. $ax \in \sum_i R t_i$. Jetzt können wir $x$ über $\{ x_1, \dots, x_n \}$ variieren lassen und erhalten, dass es ein $a \in R \setminus \{0\}$ gibt, sodass
  $$
  ax_j \in \sum_i R t_i \qquad \forall j.
  $$
  Daraus folgt $aM \subset \sum_i R t_i$. Nun erhalten wir die gewünschte Einbettung als die Komposition
  \begin{align*}
  & M \overset{\iso}{\to} aM \subset \sum_i R t_i \overset{\iso}{\to} R^d \\
  & m \mapsto am.
  \end{align*}

\end{enumerate}
\end{proof}


\subsection{Moduln über Hauptidealbereichen}

\begin{thm}
Sei $R$ ein Hauptidealbereich. Dann ist jeder Untermodul eines freies $R$-Moduls vom Rang $n$ auch frei vom Rang $\leq n$.
\end{thm}

\begin{proof}
OBDA können wir direkt mit Untermoduln von $R^n$ arbeiten. Der Beweis des Theorems ist durch die Induktion nach dem Rang gegeben.

\smallskip

\noindent {\it Induktionsanfang: } Sei $n=1$, d.h. wir befassen uns mit Untermoduln von $R$. Die Untermoduln von $R$ sind genau die Ideale von $R$ (gilt für beliebige Ringe). Da $R$ ein Hauptidealbereich ist, sind alle Ideale frei vom Rang $1$. Um das zu sehen, betrachten wir den Homomorphismus
\begin{align*}
& R \to R \\
& a \mapsto ax
\end{align*}
welcher injektiv ist, und dessen Bild das Ideal $(x)$ ist.

\smallskip
\noindent {\it Induktionsschritt: } Wir nehmen an, dass die Aussage für $R^n$ mit $n \geq 1$ bekannt ist. Zu zeigen ist, dass jeder Untermodul $M \subset R^{n+1}$ frei vom Rang $\leq n+1$ ist. Sei $\pi \colon M \to R^n$ der Homomorphismus gegeben als die Komposition $M \subset R^{n+1} = R \oplus R^n \to R^n$. Nach der Induktionsannahme ist das Bild $\pi (M) \subset R^n$ frei vom Rang $\leq n$. Somit erhalten wir nach Lemma 2.16 eine Zerlegung
\begin{align*}
M  \iso N \oplus \Ker \pi,
\end{align*}
wobei
\begin{align*}
\Ker \pi = M \cap (R \oplus 0).
\end{align*}
Insbesondere ist $\Ker \pi$ ein Untermodul von $(R \oplus 0)$. Da alle Untermoduln von $R = R \oplus 0$ frei vom Rang $\leq 1$ sind, sehen wir, dass $M$ frei vom Rang $\leq n+1$ ist.
\end{proof}

\begin{remark}
Übungsblatt 5:
\begin{enumerate}
  \item Wenn $R$ kein Hauptidealbereich ist, es also ein Ideal $I \subset R$ gibt, das kein Hauptideal ist, dann ist $I$ kein freier $R$-Modul.
  \item Wenn $R$ ein Hauptidealring ist, aber einen Nullteiler $x \neq 0$ hat, dann ist das Hauptideal $(x)$ kein freier $R$-Modul.
\end{enumerate}
\end{remark}

\begin{cor}
Endlich erzeugte torsionsfreie Moduln über einem Haupidealbereich sind frei.
\end{cor}

\begin{proof}
Folgt sofort aus Lemma 2.21 und Theorem 2.22.
\end{proof}

\begin{cor}
Jeder endlich erzeugte Modul $M$ über einem Haupidealbereich $R$ lässt sich als direkte Summe
\begin{align*}
M \iso   R^n  \oplus  M_{tors}
\end{align*}
schreiben.
\end{cor}

\begin{proof}
Nach Korollar 2.24 ist $M/M_{tors}$ frei von endlichem Rang, d.h.\ $M/M_{tors} \iso R^n$. Jetzt kann man Lemma 2.16 auf die kanonische Projektion $M \to M/M_{tors}$ anwenden.
\end{proof}


\begin{thm}
Jeder endlich erzeugte Torsionsmodul $M$ (d.h.\ $M=M_{tors}$) über einem Haupidealbereich $R$ lässt sich als direkte Summe
\begin{align*}
M \iso   \bigoplus_{i=1}^s  R/(a_i).
\end{align*}
schreiben.
\end{thm}

\begin{proof}
Ohne Beweis.
\end{proof}




\subsection{Moduln über nichtkommutativen Ringen}

In diesem Abschnitt sind Ringe nicht unbedingt kommutativ.


\begin{definition}
Sei $R$ ein Ring.
\begin{itemize}
  \item Ein {\sf $R$-Linksmodul} ist eine abelsche Gruppe $M$ (additiv geschrieben) ausgestattet mit einer linken \glqq Wirkung\grqq\, von $R$
  \begin{align*}
    & \mu \colon R \times M \to M \\
    & \quad  (a,m) \mapsto \mu(a,m) = a \cdot m,
  \end{align*}
  die die folgenden Eigenschaften haben soll:
  \begin{align*}
    & (a+b) \cdot x = a \cdot x + b \cdot x \\
    & a \cdot (x+y) = a \cdot x + a \cdot y \\
    & (a \cdot b) \cdot x = a \cdot (b \cdot x) \\
    & 1 \cdot x = x
  \end{align*}
  für alle $a,b \in R$ und $x,y \in M$.

  \item Ein {\sf $R$-Rechtsmodul} ist eine abelsche Gruppe $M$ ausgestattet mit einer rechten \glqq Wirkung\grqq\, von $R$
    \begin{align*}
      & \mu \colon M \times R \to M \\
      & \quad  (m,a) \mapsto \mu(m,a) = m \cdot a,
    \end{align*}
    die die folgenden Eigenschaften haben soll:
    \begin{align*}
      &   x \cdot (a+b) = x \cdot a + x \cdot b \\
      & (x+y) \cdot a   = x \cdot a + y \cdot a \\
      &  x \cdot (a \cdot b) = (x \cdot a) \cdot b \\
      & x \cdot 1 = x
    \end{align*}
    für alle $a,b \in R$ und $x,y \in M$.
\end{itemize}
Homomorphismen, Untermoduln und Faktormoduln werden analog zum kommutativen Fall definiert.
\end{definition}


Jetzt werden wir drei Klassen von Beispielen einführen.

\subsubsection{Gruppendarstellungen}

Sei $G$ eine Gruppe (z.B.\ eine endliche Gruppe) und $k$ ein Körper.

Eine {\sf Darstellung} von $G$ ist ein Gruppenhomomorphismus $\rho \colon G \to GL(V)$, wobei $V$ ein $k$-Vektorraum ist. Mit anderen Worten eine Darstellung ist eine lineare Wirkung von $G$ auf $V$. Ein Homomorphismus von $\rho_1 \colon G \to GL(V)$ nach $\rho_2 \colon G \to GL(W)$ ist eine $k$-lineare Abbildung $\psi \colon V \to W$, die mit der Wirkung von $G$ kommutiert, d.h. es gilt $\psi \circ \rho_1 = \rho_2$.

Der {\sf Gruppenring} von $G$ über $k$ ist die Menge
\begin{align*}
  k[G]  =  \Big \{ \sum_{g \in G} a_g g \mid \text{ nur endliche viele $a_g$ sind $\neq 0$} \Big \}
\end{align*}
mit der koeffizientenweisen Addition und mit der Multiplikation der Form
\begin{align*}
  \big( \sum_{g \in G} a_g g \big) \cdot \big( \sum_{h \in G} b_h h \big) := \sum (a_g b_h) (gh).
\end{align*}
Der Gruppenring ist genau dann kommutativ, wenn die Gruppe $G$ abelsch ist.

\medskip

\noindent Darstellungen von $G$ = $k[G]$-Linksmoduln.


\subsubsection{Köcherdarstellungen}

Köcher, Köcherdarstellungen und Pfadalgebren wurden in der Vorlesung kurz eingeführt. Am besten schauen Sie noch diese Notizen von Michel Brion an.

\subsubsection{$\cD$-Moduln}

Polynomielle Differentialoperatoren endlicher Ordnung in einer Variable $x$ bilden einen nichtkommutativen Ring $\cD$. Als Menge setzen wir
\begin{align}\label{Eq.: Weyl algebra}
\cD = \Big \{ \sum_{i=0}^\infty a_i(x) \partial_x^i  \mid a_i(x) \in k[x] \text{ nur endlich viele $\neq 0$} \Big \}.
\end{align}
Die Addition:
\begin{align*}
\sum_{i=0}^\infty a_i(x) \partial_x^i + \sum_{i=0}^\infty b_i(x) \partial_x^i   = \sum_{i=0}^\infty (a_i(x) + b_i(x)) \partial_x^i.
\end{align*}
Die Multiplikation: erst formal ausmultiplizieren
\begin{align*}
\sum_{i=0}^\infty a_i(x) \partial_x^i \cdot \sum_{i=0}^\infty b_i(x) \partial_x^i = \sum a_i(x) \partial_x^i b_j(x) \partial_x^j
\end{align*}
und dann in die Form \eqref{Eq.: Weyl algebra} mit Hilfe von
\begin{align*}
[\partial_x, x] = \partial_x x - x \partial_x = 1
\end{align*}
bringen. Z.B.
$$
(x^2\partial_x + 1) \cdot x\partial_x = x^2 \partial_x x \partial_x + x\partial_x = x^2 (x \partial_x + 1) \partial_x + x\partial_x =x^3 \partial_x^2  + (x+x^2)\partial_x.
$$


\bigskip
\noindent Moduln über dem Ring $\cD$ nennt man $\cD$-Moduln. Diese und deren Verallgemeinerungen (auf mehrere Variablen etc.) spielen eine wesentliche Rolle in der Mathematik. LINK Solche Begriffe wie z.B. Vektorbündel mit Zusammenhang lassen sich auch in der Sprache von $\cD$-Moduln auffassen.

\begin{example}
\
\begin{enumerate}
\item Der Polynomring $\cO = k[x]$ mit der üblichen Ableitung ist ein $\cD$-Linksmodul.

\item Sei $P \in \cD$ ein Differentialoperator (z.B. $P = \partial_x^2 - 1$) und
\begin{align}
P y = 0
\end{align}
die dazugehörige Differentialgleichung (für $P = \partial_x^2 - 1$ heißt diese Airy--Gleichung). Man ordnet dem Operator $P$ einen $\cD$-Linksmodul zu
\begin{align*}
M = \cD/\cD P.
\end{align*}
Es ist leicht zu sehen, dass die Lösungen von (2.2) in 1-zu-1-Beziehung stehen mit Homomorphismen von $\cD$-Linksmoduln
\begin{align*}
\Hom_{\cD}(\cD/\cD P , \cO).
\end{align*}
\end{enumerate}
\end{example}




\newpage
\section{Kategorientheorie}

\subsection{Kategorien und Funktoren}

\begin{definition}
Eine {\sf Kategorie} $\cC$ ist folgendes Datum:
\begin{enumerate}
  \item {\sf Objekte}: Eine Klasse von Objekten $Obj \, \cC$ ist gegeben. Elemente von $Obj \, \cC$ werden wir mit großen lateinischen Buchstaben $X,Y,Z \dots$ bezeichnen.

  \item {\sf Morphismen}: Für jedes Paar $X,Y \in Obj \, \cC$ ist eine Menge $\Hom(X,Y)$ gegeben. Elemente von $\Hom(X,Y)$ nennt man {\sf Morphismen} von $X$ nach $Y$. Für ein $f \in \Hom(X,Y)$ kann man alternativ schreiben $f \colon X \to Y$ oder $X \overset{f}{\to} Y$.

  \item {\sf Komposition}: Für jedes Tripel $X,Y,Z \in  Obj \, \cC$ ist eine Kompositionsabbildung gegeben
  \begin{align*}
  & \Hom(X,Y) \times \Hom(Y,Z) \to \Hom(X,Z) \\
  & \hspace{60pt} (f,g) \qquad  \mapsto \qquad g \circ f.
  \end{align*}
  Die Komposition muss assoziativ sein.

  \item {\sf Identität}: Für jedes $X \in Obj \, \cC$ existiert $\id_X \in \Hom(X,X)$ mit der Eigenschaft
  \begin{align*}
  \id_X \circ f = f  \\
  g \circ \id_X = g
  \end{align*}
  für alle Morphismen $f,g$.\footnote{Die Identität ist dadurch eindeutig definiert.}
\end{enumerate}
\end{definition}

\begin{example}
Hier sind ein paar Beispiele von Kategorien, die wir schon gesehen haben.
\begin{enumerate}
  \item $Sets$ --- die Kategorie der Mengen. Objekte sind Mengen, Morphismen sind Abbildungen.
  \item $Groups$  --- die Kategorie der Gruppen. Objekte sind Gruppen, Morphismen sind Gruppenhomomorphismen.
  \item $Ab$ --- die Kategorie der abelschen Gruppen. Objekte sind abelsche Gruppen, Morphismen sind Gruppenhomomorphismen.
  \item $Rings$ --- die Kategorie der kommutativen Ringen. Objekte sind Ringe, Morphismen sind Ringhomomorphismen.
  \item $Vect_k$ --- die Kategorie der $k$-Vektorräume. Objekte sind Vektorräume über $k$, Morphismen sind $k$-lineare Abbildungen.
  \item $R \mhyphen \Mod$ --- die Kategorie der $R$-Moduln. Objekte sind $R$-Moduln, Morphismen sind Modulhomomorphismen.
  \item $Top$ --- die Kategorie der topologischen Räume. Objekte sind topologische Räume, Morphismen sind stetige Abbildungen.
  \item Sei $R$ ein Ring (nicht unbedingt kommutativ). Dann kann man $R \mhyphen \Mod$ und $\Mod \mhyphen R$ definieren.
\end{enumerate}
\end{example}

\begin{definition}
Sei $\cC$ eine Kategorie und $f \in \Hom(X,Y)$ ein Morphismus.
\begin{enumerate}
  \item $f$ heißt {\sf Monomorphismus} $\iff$ aus $f \circ g_1 = f \circ g_2$ folgt $g_1 = g_2$.
  \item $f$ heißt {\sf Epimorphismus} $\iff$ aus $g_1 \circ f = g_2 \circ f$ folgt $g_1 = g_2$.
  \item $f$ heißt {\sf Isomorphismus} $\iff$ $f$ ist invertierbar, d.h. es existiert $g \in \Hom(Y,X)$, sodass
  \begin{align*}
  g \circ f = id_X \text{  und  } f \circ g = \id_Y.
  \end{align*}

  \item $\End(X):= \Hom (X,X)$.
  \item $Aut(X)=$  invertierbare Endomorphismen.
\end{enumerate}
\end{definition}

\begin{remark}
Es folgt sofort, dass ein Isomorphismus immer ein Monomorphismus und Epimorphismus ist. Die Umkehrung ist aber falsch ($\bZ \to \bQ$ in $Rings$).
\end{remark}

\begin{definition}
Sei $\cC$ eine Kategorie und $X,Y \in Obj \, \cC$ zwei Objekte. Ein {\sf Produkt} von $X$ und $Y$ ist ein Objekt $T$ von $\cC$ ausgestattet mit zwei Morphismen $T \to X$ und $T \to Y$, sodass die folgende universelle Eigenschaft erfüllt ist: für jedes Objekt $Z$ ausgestattet mit Morphismen $Z \to X$ und $Z \to Y$ existiert ein eindeutiger Morphismus $\psi \colon Z \to T$, sodass das Diagramm
\begin{align*}
  \xymatrix{
   Z \ar@{-->}[rr]^{\exists ! \psi } \ar[dr] \ar[drrr] &&   T \ar[dl] \ar[dr]\\
   & X & & Y
  }
\end{align*}
kommutiert.

Wenn es existiert, ist ein Produkt eindeutig bis auf eine eindeutige Isomorphie definiert und wird mit $X \times Y$ bezeichnet.
\end{definition}



\begin{definition}
Sei $\cC$ eine Kategorie und $X,Y \in Obj \, \cC$ zwei Objekte. Ein {\sf Koprodukt} von $X$ und $Y$ ist ein Objekt $T$ von $\cC$ ausgestattet mit Morphismen $X \to T$ und $Y \to T$, sodass die folgende universelle Eigenschaft erfüllt ist: für jedes Objekt $Z$ ausgestattet mit Morphismen $X \to Z$ und $Y \to Z$ existiert ein eindeutiger Morphismus $\psi \colon T \to Z$, sodass das Diagramm
\begin{align*}
  \xymatrix{
   Z   &&   T  \ar@{-->}[ll]_{\exists ! \psi }  \\
   & X \ar[lu] \ar[ur] & & Y \ar[ulll] \ar[ul]
  }
\end{align*}
kommutiert.

Wenn es existiert, ist ein Koprodukt eindeutig bis auf eine eindeutige Isomorphie definiert und wird mit $X \coprod Y$ bezeichnet.
\end{definition}


\begin{definition}
Seien $\cC$ und $\cD$ zwei Kategorien. Ein {\sf kovarianter Funktor} $F \colon \cC \to \cD$ ist folgendes Datum:
\begin{enumerate}
\item Eine Abbildung $Obj\, \cC \to Obj\, \cD$, $X \mapsto F(X)$;
\item Für jedes paar $X,Y \in Obj\, \cC$ eine Abbildung $\Hom_{\cC}(X,Y) \to \Hom_{\cD}(F(X),F(Y))$, sodass
  \begin{align*}
  & F(\id_X) = \id_{F(X)} \\
  & F(\varphi \circ \psi) = F(\varphi) \circ F(\psi).
  \end{align*}
\end{enumerate}

\noindent Ein {\sf kontravarianter Funktor} $F \colon \cC \to \cD$ ist folgendes Datum:
\begin{enumerate}
  \item Eine Abbildung $Obj \cC \to Obj \cD$, $X \mapsto F(X)$;
  \item Für jedes paar $X,Y \in Obj \cC$ eine Abbildung $\Hom_{\cC}(X,Y) \to \Hom_{\cD}(F(Y),F(X))$, sodass
  \begin{align*}
  & F(\id_X) = \id_{F(X)} \\
  & F(\varphi \circ \psi) =  F(\psi) \circ F(\varphi).
  \end{align*}
\end{enumerate}
\end{definition}

\begin{example}
\
\begin{enumerate}
  \item $\Id_{\cC} \colon \cC \to \cC$
  \item $\Hom(T,\blank)  \colon \cC \to Sets$

  \item Vergissfunktoren: $Top \to Sets$, $R \mhyphen \Mod \to Ab \to Sets$

  \item $GL_n \colon Rings \to Groups$
  \item $SL_n \colon Rings \to Groups$
  \item $V \mapsto V^*$ ist kontravariant.
  \item $\Spec \colon Rings^{op} \to Top$
\end{enumerate}
\end{example}

\begin{remark}
Gegeben eine Kategorie $\cC$ definiert man die duale Kategorie $\cC^{op}$ wie folgt: $Obj\, \cC^{op} = Obj\, \cC$ und $\forall X,Y \in \cC^{op}$ definieren wir $\Hom_{\cC^{op}}(X,Y)=\Hom_{\cC}(Y,X)$.

Es ist leicht zu sehen, dass es eine Bijektion zwischen kontravarianten Funktoren $\cC \to \cD$ und kovarianten Funktoren $\cC^{op} \to \cD$ gibt.
\end{remark}


\begin{definition}
Seien $F,G \colon \cC \to \cD$ Funktoren. Eine natürliche Transformation (oder Morphismus von Funktoren) $\rho \colon F \to G$ ist folgendes Datum:
\begin{enumerate}
  \item Für jedes $X \in \cC$ ein Morphismus $F(X) \overset{\rho(X)}{\to} G(X)$;
  \item Für jeden Morphismus $\varphi \colon X \to Y$ soll das Diagramm
  \begin{align*}
  \xymatrix{
  F(X) \ar[r]^{\rho(X)} \ar[d]_{F(\varphi)} & G(X) \ar[d]^{G(\varphi)} \\
  F(Y) \ar[r]^{\rho(Y)} & G(Y)
  }
  \end{align*}
  kommutieren.
\end{enumerate}
\end{definition}

\begin{remark}
$Funct(\cC, \cD)$ ist eine Kategorie (Übungsblatt 7).
\end{remark}

\begin{example}
$SL_n \to GL_n \to GL_1$
\end{example}

Man kann von Isomorphismen von Kategorien sprechen, aber dieser Begriff ist nicht besonders nützlich. Stattdessen führt man Kategorienäquivalenzen ein, welche sehr oft benutzt werden.
\begin{definition}
Ein Funktor $F\colon \cC \to \cD$ heißt {\sf Äquivalenz}, wenn es einen Funktor $G \colon \cD \to \cC$ gibt, sodass
\begin{align*}
& F \circ G \iso \Id_{\cD} \\
& G \circ F \iso \Id_{\cC} \\
\end{align*}
\end{definition}


\begin{example}
Sei $Vect_k^{n}$ die Kategorie der $n$-dimensionalen $k$-Vektorräume und sei $\cC$ die Kategorie mit einem einzelnen Objekt $k^n$ und linearen Abbildung $\Hom_k(k^n,k^n)$ als Morphismen. Der natürlich Inklusionsfunktor $\cC \to Vect_k^{n}$ ist eine Kategorienäquivalenz. \end{example}



\subsection{Exakte Sequenzen}

Sei $R$ ein Ring.

\begin{definition}
Eine Sequenz von $R$-Moduln und Homomorphismen
\begin{align*}
\dots \to M_{i-1} \overset{f}{\to} M_i \overset{g}{\to} M_{i+1} \to \dots
\end{align*}
heißt {\sf exakt am $M_i$}, wenn gilt $\Im f = \Ker g$. Die Sequenz heiß {\sf exakt}, wenn sie an jeder Stelle exakt ist.

Insbesondere gilt:
\begin{enumerate}
  \item $0 \to M' \overset{f}{\to} M$ ist exakt $\iff$  $f$ ist injektiv
  \item $M \overset{g}{\to} M'' \to 0$ ist exakt $\iff$  $g$ ist surjektiv
  \item Eine Sequenz
  $$
  0 \to M' \overset{f}{\to} M \overset{g}{\to} M'' \to 0
  $$
  ist exakt $\iff$  $f$ ist injektiv, $g$ ist surjektiv und gilt
  $$
  \Coker(f)=M/\Im f \iso M''.
  $$
  Solche exakte Sequenzen nennt man {\sf kurze exakte Sequenzen}.\footnote{kurze exakte Sequenz = k.e.S.}
\end{enumerate}
\end{definition}

\begin{example}
\
\begin{enumerate}
\item Gegeben ein $R$-Modul $M$ und ein Untermodul $M' \subset M$, kann man immer die folgende k.e.S. betrachten
$$
0 \to M' \to M  \to M/M' \to 0.
$$
\item Gegeben zwei $R$-Moduln $M$ und $N$ kann man immer die folgende k.e.S. betrachten
$$
0 \to M \overset{f}{\to} M \oplus N \overset{g}{\to} N \to 0,
$$
wobei $f(m) = (m,0)$ und $g(m,n)=n$.

\item Man sagt, dass eine kurze exakte Sequenz
$$
0 \to M' \overset{f}{\to} M \overset{g}{\to} M'' \to 0
$$
{\sf spaltet}, wenn $g$ einen Schnitt hat. D.h. es existiert $h \colon M'' \to M$ mit $g \circ h = \id_{M''}$. In diesem Fall ist die k.e.S. isomorph zu
$$
0 \to M' \overset{f}{\to} f(M') \oplus h(M'') \overset{g}{\to} M'' \to 0.
$$

\item Die k.e.S. $0 \to \bZ \to \bZ \to \bZ / 2 \bZ \to 0$ spaltet nicht.

\item Sei $R$ ein Hauptidealbereich und $M$ ein endlich erzeugter $R$-Modul. Die folgende k.e.S.
$$
0 \to M_{tors} \to M \to M/M_{tors} \to 0
$$
spaltet immer, da nach Korollar 2.24 $M/M_{tors}$ frei ist (vgl. Lemma 2.16).
\end{enumerate}
\end{example}

\begin{proposition}\label{prop.: hom-is-left-exact}
Sei $N$ ein $R$-Modul und
$$
0 \to M' \overset{f}{\to} M \overset{g}{\to} M'' \to 0
$$
eine kurze exakte Sequenz. Dann sind die induzierten Sequenzen
\begin{align*}
&  0 \to \Hom(N,M') \to \Hom(N,M) \to \Hom(N,M'') \\
&  0 \to \Hom(M'', N) \to \Hom(M,N) \to \Hom(M',N)
\end{align*}
exakt.
\end{proposition}

\begin{proof}
Übung.
\end{proof}


\begin{remark}
Aus Satz \ref{prop.: hom-is-left-exact} folgt, dass die Funktoren
\begin{align*}
& \Hom(N,\blank)  \colon R \mhyphen \Mod \to R \mhyphen \Mod \\
& \Hom(\blank, N)  \colon (R \mhyphen \Mod)^{op} \to R \mhyphen \Mod
\end{align*}
linksexakt sind.
\end{remark}


\subsection{Additive und abelsche Kategorien}

\begin{definition}
Eine Kategorie $\cC$ heißt {\sf additiv}, wenn sie die folgende Eigenschaften hat:
\begin{enumerate}
  \item $\forall$ $X,Y \in \cC$ ist die Menge $\Hom_{\cC}(X,Y)$ mit der Struktur einer abelschen Gruppe ausgestattet (insbesondere $\Hom_{\cC}(X,Y) \neq \emptyset$, da wir immer $0 \in \Hom_{\cC}(X,Y)$ haben), und $\forall$ $X,Y,Z \in \cC$ ist die Kompositionsabbildung
  \begin{align*}
  \Hom_{\cC}(X,Y) \times \Hom_{\cC}(Y,Z) \to \Hom_{\cC}(X,Z)
  \end{align*}
  {\sf biadditiv}, d.h. $\forall \,  f,g,h$ gilt $(f+g) \circ h  = f \circ h + g \circ h$ und $h \circ (f+g) = h \circ f + h \circ g$.

  \item Es existiert ein Objekt $\textbf{0}$, Nullobjekt genannt, sodass $\Hom_{\cC}(X,\textbf{0}) = \{0\} = \Hom_{\cC}(\textbf{0},X)$ $\forall\, X$, d.h. es gibt genau einen Morphismus von/nach $\textbf{0}$.

  \item $\forall$ $X,Y \in \cC$ existiert das Produkt $X \times Y$.
\end{enumerate}
\end{definition}



\begin{example}
$Ab, R \mhyphen \Mod, Vect_k$ sind additive Kategorien.
\end{example}

\medskip

\noindent Sei $\cC$ eine additive Kategorie und $X, Y \in \cC$. Mit Hilfe der universellen Eigenschaft von $X \times Y$ kann man einen Morphismus $X \to X \times Y$ definieren (betrachte $X \overset{\id_X}{\to} X$ und $X \overset{0}{\to} Y$). Analog kann man $Y \to X \times Y$ definieren.


\begin{lemma}
Mit den obigen Morphismen $X \to X \times Y$ und $Y \to X \times Y$ hat $X \times Y$ die universelle Eigenschaft des Koprodukts.
\end{lemma}

\begin{proof}
Übungsblatt 8.
\end{proof}



\begin{definition}
Sei $\cA$ eine additive Kategorie und $f \colon X \to Y$ ein Morphismus.
\begin{enumerate}
  \item Ein Objekt $T$ ausgestattet mit einem Morphismus $h \colon T \to X$ mit $f\circ h = 0$ heißt {\sf Kern} von $f$, wenn die folgende universelle Eigenschaft erfüllt ist: für jedes $\varphi \colon  Z \to X$ mit $f\circ \varphi = 0$ existiert ein eindeutiger $\psi \colon Z \to T$ mit $\varphi = h \circ \psi$.
  \begin{align*}
  \xymatrix{
  Z \ar[dr]^{\varphi} \ar@{-->}[d]_{\exists ! \psi} \\
  T \ar[r]^h & X \ar[r]^f & Y
  }
  \end{align*}
  Wenn ein Kern $T \overset{h}{\to} X$ existiert, ist dieser eindeutig bis auf eine eindeutige Isomorphie definiert und mit $\Ker(f)$ bezeichnet.


  \item  Ein Objekt $T$ ausgestattet mit einem Morphismus $h \colon Y \to T$ mit $h \circ f = 0$, heißt {\sf Kokern} von $f$, wenn die folgende universelle Eigenschaft erfüllt ist: für jedes $\varphi \colon  Y \to Z$ mit $\varphi \circ f= 0$ existiert ein eindeutiger $\psi \colon T \to Z$ mit $\varphi = \psi \circ h$.
  \begin{align*}
  \xymatrix{
  X \ar[r]^f & Y \ar[r]^h \ar[dr]_{\varphi} & T \ar@{-->}[d]^{\exists ! \psi} \\
   &&Z
  }
  \end{align*}

  Wenn ein Kokern $Y \overset{h}{\to} T$ existiert, ist dieser eindeutig bis auf eine eindeutige Isomorphie definiert und mit $\Coker(f)$ bezeichnet.


  \item Das {\sf Bild} von $f$ wird als Kern vom Kokern definiert: $\Im f = \Ker(\Coker f)$.

  \item Das {\sf Kobild} von $f$ wird als Kokern vom Kern definiert: $\Coim f = \Coker(\Ker f)$.
  \end{enumerate}
\end{definition}

\bigskip

\noindent Sei $\cA$ eine additive Kategorie und $\varphi \colon X \to Y$ ein Morphismus. Dann kann man die folgende Sequenz von Objekten und Morphismen betrachten
\begin{align*}
\Ker \varphi \overset{a}{\to} X \overset{b}{\to} \Coim \varphi \overset{c}{\to} \Im \varphi \overset{d}{\to} Y \overset{e}{\to} \Coker \varphi
\end{align*}
mit der Eigenschaft $d \circ c \circ b = \varphi$. Hier sind $a,b,d,e$ die Morphismen, die in den Definitionen von (Ko-)Kern und (Ko-)Bild vorkommen, und der Morphismus $\Coim \varphi \overset{c}{\to} \Im \varphi$ entsteht auf eine kanonische Weise (Übung). In solchen Kategorien wie $Ab$ oder $R \mhyphen \Mod$ ist $\Coim \varphi \overset{c}{\to} \Im \varphi$ immer ein Isomorphismus, aber nicht in beliebigen additiven Kategorien. Dies führt uns zur nächsten Definition.



\begin{definition}
Eine additive Kategorie $\cA$ heißt {\sf abelsch}, wenn:
\begin{enumerate}
  \item in $\cA$ beliebige Kerne und Kokerne existieren;
  \item für jedes $\varphi$ der kanonische Morphismus $\Coim \varphi \overset{c}{\to} \Im \varphi$ ein Isomorphismus ist.
\end{enumerate}




\end{definition}

\begin{example}
\
\begin{enumerate}
  \item $Ab, R \mhyphen \Mod$ sind abelsch.
  \item Die Kategorie der freien abelschen Gruppen ist nicht abelsch! Für den Morphismus $\bZ \overset{\cdot 2}{\to} \bZ$ existieren zwar der Kern und der Kokern, aber der kanonische Morphismus $\Coim \varphi \to \Im \varphi$ ist kein Isomorphismus!
\end{enumerate}
\end{example}

\begin{definition}
Seien $\cA$ und $\cB$ additive Kategorien. Ein Funktor $F \colon \cA \to \cB$ heißt {\sf additiv}, wenn die Abbildung
$\Hom_{\cA}(X,Y) \to \Hom_{\cB}(F(X), F(Y))$ ein Homomorphismus von abelschen Gruppen ist.
\end{definition}

\noindent Der Begriff von exakten Sequenzen lässt sich auf allgemeine abelsche Kategorien übertragen.

\begin{definition}
Seien $\cA$, $\cB$ abelsche Kategorien und $F \colon \cA \to \cB$ ein additiver Funktor.
\begin{enumerate}
  \item $F$ heißt {\sf exakt}, wenn für jede k.e.S.  $0 \to X \to Y \to Z \to 0$ in $\cA$
 die induzierte Sequenz
 $$
 0 \to F(X) \to F(Y) \to F(Z) \to 0
 $$
 exakt in $\cB$ ist.

 \item $F$ heißt {\sf linksexakt}, wenn für jede k.e.S. $0 \to X \to Y \to Z \to 0$ in $\cA$
 die induzierte Sequenz
 $$
 0 \to F(X) \to F(Y) \to F(Z)
 $$
 exakt in $\cB$ ist.

 \item $F$ heißt {\sf rechtsexakt}, wenn für jede k.e.S. $0 \to X \to Y \to Z \to 0$ in $\cA$
 die induzierte Sequenz
 $$
 F(X) \to F(Y) \to F(Z) \to 0
 $$
 exakt in $\cB$ ist.
\end{enumerate}
\end{definition}
















\newpage

\section{Tensorprodukt}

\subsection{Definition und erste Eigenschaften}

\begin{definition}
Seien $M,N,P$ Moduln über einem Ring $R$. Eine Abbildung
\begin{align*}
f \colon M \times N \to P
\end{align*}
heißt {\sf bilinear}, wenn gilt
\begin{align*}
& f(x+y, z) = f(x,z) + f(y,z) \\
& f(ax, z) = af(x,y) \\
& f(z, x+y) = f(z, x) + f(z,y) \\
& f(z,ax) = af(z,x).
\end{align*}
\end{definition}

\begin{proposition}
Seien $M,N$ zwei $R$-Moduln. Es existiert ein $R$-Modul $T$ und eine $R$-bilineare Abbildung $g \colon M \times N \to T$, sodass die folgende universelle Eigenschaft erfüllt ist: für jeden $R$-Modul $P$ und jede $R$-bilineare Abbildung $f \colon M \times N \to P$ existiert ein eindeutiger Homomorphismus von $R$-Moduln $\psi \colon T \to P$, sodass das Diagramm
\begin{align*}
\xymatrix{
M \times N \ar[r]^g \ar[rd]_f & T  \ar@{-->}[d]^{\exists ! \psi} \\
           & P
}
\end{align*}
kommutiert. Das Paar $g \colon M \times N \to T$ ist eindeutig bis auf eindeutige Isomorphie definiert.
\end{proposition}

\begin{proof}
Die Eindeutigkeit folgt sofort aus der universellen Eigenschaft.

\smallskip

\noindent {\it Existenz:} Wir geben eine explizite Konstruktion von $T$. Sei $C$ der freie $R$-Modul mit der Basis $(x,y)$ mit $x,y \in M \times N$, d.h.
\begin{equation*}
C := \bigoplus_{x,y \in M \times N} R (x,y).
\end{equation*}
Nun sei $D \subset C$ ein Untermodul von $C$ erzeugt durch
\begin{align}\label{eq.: relations tensor product} \notag
& (x+x',y) - (x,y) - (x',y) \\ \notag
& (x,y+y') - (x,y) - (x,y') \\
& (ax,y) - a(x,y) \\ \notag
& (x,ay) - a(x,y).
\end{align}
Jetzt definiere
\begin{align*}
T:=C/D.
\end{align*}
Die natürliche Abbildung $g$
\begin{align*}
& M \times N \to C \to T\\
& (x,y) \mapsto (x , y) \mapsto x \otimes y
\end{align*}
ist bilinear. Jetzt bleibt uns nur die universelle Eigenschaft für $g \colon M \times N \to T$ zu überprüfen.

Sei $f \colon M \times N \to P$ eine bilineare Abbildung. Erst definieren wir einen Homomorphismus $C \to P$, indem wir jedes Basiselement $(x,y)$ nach $f(x,y)$ abbilden. Da $f$ bilinear ist, liegen die Elemente \eqref{eq.: relations tensor product} im Kern. Jetzt erhalten wir nach Homomorphiesatz $\psi \colon T \to P$.
\end{proof}

\begin{remark}
\
\begin{enumerate}
\item Der Modul $T$ wird mit $M \otimes_R N$ bezeichnet und {\sf Tensorprodukt} von $M$ und $N$ über $R$ genannt. Die bilineare Abbildung $g$ ist ein Teil dieser Struktur. Die explizite Konstruktion aus dem Beweis $T=C/D$ werden wir nicht mehr brauchen.

\item Das Bild von $(x,y) \in M \times N$ in $M \otimes_R N$ bezeichnen wir mit $x \otimes y$.

\item Da die Abbildung $M \times N \to M \otimes_R N$ bilinear ist, gilt
\begin{align*}
& (x+x') \otimes y = x \otimes y + x' \otimes y \\
& x \otimes (y+y') = x \otimes y + x \otimes y' \\
& (ax) \otimes y = a (x \otimes y) \\
& x \otimes (ay) = a (x \otimes y)
\end{align*}

\item Als $R$-Modul ist $M \otimes_R N$ von den Elementen der Form $x \otimes y$ erzeugt, d.h. jedes Element in $M \otimes_R N$ ist eine endliche Summe der Form $\sum_i x_i \otimes y_i$.

\item $x \otimes 0 = 0 \otimes x =0$.

\item In $\bZ \otimes \bZ / 2 \bZ$ gilt $2 \otimes \bar{1} = 0$, aber nicht in $2\bZ \otimes \bZ / 2 \bZ$.

\end{enumerate}
\end{remark}

\begin{example}
\
\begin{enumerate}
\item Sei $k$ ein Körper und $V, W$ endlich dimensionale $k$-Vektorräume. Wenn $e_1, \dots , e_n$ und $f_1, \dots, f_m$ Basen von $V$ und $W$ sind. Dann ist $\{e_i \otimes f_j\}$ eine Basis von $V \otimes_k W$. Insbesondere haben wir
$$
\dim_k \big( V \otimes_k W \big) = \dim_k V \cdot \dim_k W.
$$

\item Es gilt $\bZ / 2 \bZ \otimes_{\bZ} \bZ / 3 \bZ \iso 0 $. Wir haben
\begin{align*}
&2 ( a \otimes b ) = 0 = 3( a \otimes b ) \quad \forall a \in \bZ / 2\bZ  , \,  \forall b \in \bZ / 3 \bZ
\end{align*}
Somit ist jedes Element von $\bZ / 2 \bZ \otimes_{\bZ} \bZ / 3 \bZ$ gleich Null.

\item Allgemeiner gilt dies $\bZ / m \bZ \otimes_{\bZ} \bZ / n \bZ \iso 0$, wenn $m$ und $n$ koprim sind (Übungsblatt 9).

\item $\bZ / 2 \bZ \otimes_{\bZ} \bZ / 2 \bZ \iso \bZ / 2 \bZ $ (Übungsblatt 9).

\item $\bZ / m \bZ \otimes_{\bZ} \bZ / n \bZ \iso \bZ / gcd(m,n) \bZ $ (Übungsblatt 9).

\item $R / I \otimes_{R} R / J \iso R / I + J$ (Übungsblatt 9).

\item $\bZ / n \bZ  \otimes_{\bZ} \bQ = 0$.

\item Allgemeiner gilt $M \otimes_{R} K = 0$ für einen Torsionsmodul $M$ über einem Integritätsbereich $R$ und $K=Quot(R)$.
\end{enumerate}
\end{example}


\begin{proposition}
Seien $M_1, \dots , M_n$ Moduln über $R$. Es existiert ein $R$-Modul $T$ und eine $R$-multilineare Abbildung $g \colon M_1  \times \dots \times M_n \to T$, sodass die folgende universelle Eigenschaft erfüllt ist: für jeden $R$-Modul $P$ und jede $R$-multilineare Abbildung $f \colon M_1 \times  \dots \times M_n \to P$ existiert ein eindeutiger Homomorphismus von $R$-Moduln $\psi \colon T \to P$, sodass das Diagramm
\begin{align*}
\xymatrix{
M_1 \times \dots \times M_n \ar[rr]^g \ar[rrd]_f & & T  \ar@{-->}[d]^{\exists ! \psi} \\
           & & P
}
\end{align*}
kommutiert. Das Paar $g \colon M_1 \times \dots \times M_n \to T$ ist eindeutig bis auf eine eindeutige Isomorphie definiert.
\end{proposition}

\begin{proposition}\label{satz-properties-of-tensor-product}
Seien $M,N,P$ Moduln über $R$. Es existieren Isomorphismen von $R$-Moduln
\begin{enumerate}
  \item $M \otimes N \cong N \otimes M$
  \item $(M \otimes N) \otimes P \cong M \otimes (N \otimes P) \cong M \otimes N \otimes P$
  \item $(M \oplus N) \otimes P \cong (M \otimes P) \oplus  (N \otimes P)$
  \item $R \otimes M \cong M$
\end{enumerate}
gegeben durch
\begin{enumerate}
  \item $x \otimes y \mapsto y \otimes x$
  \item $(x \otimes y) \otimes z  \mapsto  x \otimes (y \otimes z)  \mapsto   x \otimes y \otimes z$
  \item $(x, y) \otimes z  \mapsto (x \otimes z , y \otimes z)$
  \item $a \otimes x \mapsto ax$.
\end{enumerate}
\end{proposition}

\begin{proof}
Siehe [AM] und Übungsblatt 9.
\end{proof}

\begin{definition}
Seien $f \colon M \to M'$ und $g \colon N \to N'$ Homomorphismen von $R$-Moduln. Definiere einen Homomorphismus von $R$-Moduln
\begin{align*}
& f \otimes g \colon M \otimes N \to M' \otimes N' \\
& (f \otimes g) (x \otimes y) = f(x) \otimes g(y).
\end{align*}

\end{definition}


\begin{definition}
Sei $R$ ein Ring und $N$ ein $R$-Modul. Dann definiert man einen Funktor
\begin{align*}
\blank \otimes_R N \colon R \mhyphen \Mod \quad &\longrightarrow  \quad R \mhyphen \Mod \\
\qquad \qquad M  \quad  &\longmapsto  \quad   M \otimes_R N \\
\qquad M_1 \overset{f}{\to} M_2  \quad&\longmapsto \quad  M_1 \otimes_R N \overset{f \otimes \id_N}{\to} M_2 \otimes_R N
\end{align*}
\end{definition}




\subsection{Restriktion und Erweiterung der Skalare}



\begin{definition}
Sei $f \colon R \to S$ ein Ringhomomorphismus.
\begin{enumerate}
\item Auf jedem $S$-Modul $N$ kann man eine $R$-Modulstruktur definieren als
$$
a x = f(a)x.
$$
Betrachtet als $R$-Modul bezeichnen wir $N$ mit ${}_RN$. Dadurch erhalten wir einen Funktor
\begin{align*}
& S \mhyphen \Mod \to R \mhyphen \Mod \\
& \quad N  \quad \mapsto \quad  {}_RN,
\end{align*}
welcher {\sf Restriktion der Skalare} genannt wird.

\item Das Tensorprodukt mit $S$ über $R$ liefert einen Funktor
\begin{align*}
& \blank \otimes_R S: R \mhyphen \Mod \to S \mhyphen \Mod \\
& \hspace{70pt} M \mapsto M_S:=M \otimes_R S,
\end{align*}
welcher {\sf Erweiterung der Skalare} genannt wird.
\end{enumerate}
\end{definition}

\begin{lemma}
Sei $f \colon R \to S$ ein Ringhomomorphismus.
\begin{enumerate}
\item Sei $N$ ein endlich erzeugter $S$-Modul. Wenn $S$ endlich erzeugt als Modul über $R$ ist, dann ist~${}_RN$ auch ein endlich erzeugter $R$-Modul.

\item Wenn $M$ ein endlich erzeugter $R$-Modul ist, ist $M_S$ endlich erzeugt als $S$-Modul.
\end{enumerate}
\end{lemma}

\begin{proof}
\
\begin{enumerate}
\item Seien $x_1, \dots , x_n \in N$ beliebige Erzeuger von $N$ über $S$ und $s_1, \dots , s_m \in S$ beliebige Erzeuger von $S$ über $R$. Dann wird $N$ von den Elementen $s_ix_j$ als $R$-Modul erzeugt.


\item Seien $x_1, \dots , x_n \in M$ beliebige Erzeuger von $M$ über $R$. Dann erzeugen $x_1 \otimes 1, \dots , x_n \otimes 1$ den Modul $M \otimes_R S$ als $S$-Modul.
\end{enumerate}
\end{proof}

\begin{lemma}
Sei $f \colon R \to S$ ein Ringhomomorphismus. Es gibt einen kanonischen Isomorphismus von $R$-Moduln
\begin{align*}
\Hom_S(M_S, N) \cong \Hom_R(M , {}_R N).
\end{align*}
\end{lemma}

\begin{proof}
Übungsblatt 9.
\end{proof}

\begin{remark}
In der Sprache der Kategorientheorie bedeutet das obige Lemma, dass der Skalarerweiterungsfunktor linksadjungiert zum Skalarrestriktionsfunktor ist.
\end{remark}


\subsection{Exaktheit des Tensorproduktes}

\begin{proposition}
Seien $M,N,P$ Moduln über $R$. Es gibt einen kanonischen Isomorphismus von $R$-Moduln
\begin{align}\label{eq.: tensor-hom-adjunction}
\Hom_R(M \otimes_R N , P) \cong \Hom_R(M,\Hom_R(N,P)).
\end{align}
\end{proposition}

\begin{proof}
Die Bijektion \eqref{eq.: tensor-hom-adjunction} ist als Komposition der Bijektion
\begin{align*}
& \Hom_R(M \otimes_R N , P) \overset{\cong}{\to}  \{R\text{-bilineare Abbildungen } M \times N \to P \},
\end{align*}
welche aus der universellen Eigenschaft des Tensorproduktes folgt, und der Bijektion
\begin{align*}
& \{R\text{-bilineare Abbildungen } M \times N \to P \} \overset{\cong}{\to} \Hom_R(M,\Hom_R(N,P)) \\
& \hspace{100pt} \varphi \hspace{100pt} \mapsto \hspace{20pt} x \mapsto \varphi(x, \blank)
\end{align*}
gegeben.
\end{proof}


\begin{remark}
In der Sprache der Kategorientheorie bedeutet Isomorphismus \eqref{eq.: tensor-hom-adjunction}, dass der Funktor $\blank \otimes_R N$ linksadjungiert zum Funktor $\Hom_R(N,\blank)$ ist.
\end{remark}

Gleich werden wir das folgende Lemma brauchen, das eine etwas stärkere Version von Satz \ref{prop.: hom-is-left-exact} ist.

\begin{lemma}\label{lemma: hom-is-left-exact-iff}
\
\begin{enumerate}
\item Die Sequenz von $R$-Moduln
\begin{align*}
M' \to M \to M'' \to 0
\end{align*}
ist genau dann exakt, wenn für jeden $R$-Modul $P$ die Sequenz
\begin{align*}
0 \to \Hom_R(M'',P) \to \Hom_R(M,P) \to \Hom_R(M',P)
\end{align*}
exakt ist.

\item Die Sequenz von $R$-Moduln
\begin{align*}
0 \to M' \to M \to M''
\end{align*}
ist genau dann exakt, wenn für jeden $R$-Modul $P$ die Sequenz
\begin{align*}
0 \to \Hom_R(P, M') \to \Hom_R(P, M) \to \Hom_R(P,M'')
\end{align*}
exakt ist.
\end{enumerate}
\end{lemma}

\begin{proof}
Übungsblatt 9.
\end{proof}


\begin{proposition}
Sei $R$ ein Ring und $N$ ein $R$-Modul. Der Funktor
\begin{align*}
& \blank \otimes_R N \colon R \mhyphen \Mod \to R \mhyphen \Mod \\
& \hspace{45pt} M \mapsto M \otimes_R N
\end{align*}
ist rechtsexakt.
\end{proposition}

\begin{proof}
Sei $N$ ein $R$-Moduln und
\begin{align}\label{eq.: exact-sequence-proof-1}
0 \to M' \overset{f}{\to}  M \overset{g}{\to} M'' \to 0
\end{align}
eine k.e.S. von $R$-Moduln. Zu zeigen ist, dass die induzierte Sequenz
\begin{align*}
M' \otimes_R N \to  M \otimes_R N \to M'' \otimes_R N \to 0
\end{align*}
exakt ist. Dafür betrachten wir einen beliebigen $R$-Modul $P$ und wenden den Funktor $\Hom_R(\blank, \Hom_R(N,P))$ auf \eqref{eq.: exact-sequence-proof-1} an. Dadurch erhalten wir eine exakte Sequenz
\begin{align}\label{eq.: exact-sequence-proof-2}
0 \to \Hom_R(M'', \Hom_R(N,P)) \to \Hom_R(M, \Hom_R(N,P)) \to \Hom_R(M', \Hom_R(N,P)),
\end{align}
da nach Satz \ref{prop.: hom-is-left-exact} der $\Hom$-Funktor linksexakt ist. Jetzt wenden wir \eqref{eq.: tensor-hom-adjunction} an \eqref{eq.: exact-sequence-proof-2} an und bekommen eine exakte Sequenz
\begin{align}\label{eq.: exact-sequence-proof-3}
0 \to \Hom_R(M'' \otimes N, P) \to \Hom_R(M \otimes N, P) \to \Hom_R(M' \otimes N, P).
\end{align}
Jetzt folgt die Aussage aus Lemma \ref{lemma: hom-is-left-exact-iff}.
\end{proof}


\begin{example}
Betrachten wir die k.e.S. $0 \to \bZ \overset{\cdot 2}{\to} \bZ \to \bZ/2 \bZ \to 0$ und tensorieren diese mit $\bZ/2 \bZ$. Dann bekommen wir die Sequenz
\begin{align*}
0 \to \bZ/2\bZ \overset{0}{\to} \bZ/2\bZ \to \bZ/2\bZ \otimes_{\bZ} \bZ/2\bZ \to 0,
\end{align*}
die keine k.e.S. ist! Also ist $\blank \otimes_{\bZ} \bZ/2\bZ$ kein exakter Funktor. Wir bekommen aber daraus den Isomorphismus $\bZ/2\bZ \otimes_{\bZ} \bZ/2\bZ \iso \bZ/2\bZ$.
\end{example}


\medskip
Da der Funktor $\blank \otimes_R N$ nicht für jeden $R$-Modul $N$ exakt ist, führt man den folgenden Begriff ein.
\begin{definition}
Ein $R$-Modul $N$ heißt {\sf flach}, wenn der Funktor $\blank \otimes_R N$ exakt ist.
\end{definition}

\begin{example}
$M = R^n$ ist flach.
\end{example}


\subsection{Tensorprodukt von Algebren}

\begin{definition}
Ein Ring $S$ ausgestattet mit einem Ringhomomorphismus $R \to S$ heißt {\sf $R$-Algebra}.
\end{definition}

\begin{remark}
Gegeben eine $R$-Algebra $R \to S$, kann man $S$ als $R$-Modul betrachten:
$$
r \cdot s := f(r)s.
$$
Umgekehrt: Gegeben ein $R$-Modul $M$, welcher zusätzlich mit einer $R$-bilinearen Multiplikation $M \times M \to M$ ausgestattet ist (assoziativ, kommutativ und mit Eins), kann man einen Ringhomomorphismus $R \to M$ definieren
\begin{align*}
& R \to M \\
& r \mapsto r1_M
\end{align*}
und eine $R$-Algebra Struktur auf $M$ erhalten.
\end{remark}

\begin{definition}
Seien $f \colon R \to S$ und $g \colon R \to T$ Algebren über $R$. Die Multiplikation
\begin{align*}
& (S \otimes_R T) \times (S \otimes_R T) \to (S \otimes_R T) \\
&  \quad  \left( (a \otimes b), (c \otimes d) \right) \quad \mapsto \quad ac \otimes bd
\end{align*}
zusammen mit dem Ringhomomorphismus
\begin{align*}
& R \to S \otimes_R T \\
& r \mapsto f(r) \otimes g(r)
\end{align*}
stattet $S \otimes_R T$ mit der Struktur einer $R$-Algebra aus.
\end{definition}

\begin{example}
\
\begin{enumerate}
\item Seien $R=k$, $S=k[x]$, $T=k[y]$. Dann gilt $k[x] \otimes_k k[y] \iso k[x,y]$.

\item $k[x] \otimes_{k[x,y]} k[y] \iso k$ (Übungsblatt 9).

\item $(R/I) \otimes_R (R/J) \iso R/(I+J)$ (Übungsblatt 9).
\end{enumerate}
\end{example}





\newpage
\section{Lokalisierung}


\subsection{Lokalisierung von Ringen}

\begin{definition}
Sei $A$ ein Ring. Eine Teilmenge $S \subset A$ heißt {\sf multiplikativ abgeschlossen}, wenn gilt
$$
S \cdot S \subset S \quad \text{und} \quad 1 \in S.
$$
\end{definition}

\begin{definition}
Seien $A$ ein Ring und $S \subset A$ eine multiplikativ abgeschlossene Teilmenge von $A$. Definiere eine Relation auf der Menge $A \times S$ als
\begin{align*}
(a,s) \sim (b,t) \iff (at-bs)u = 0 \quad \text{für ein }  u \in S.
\end{align*}
Diese Relation ist reflexiv, symmetrisch und transitiv, und somit ist sie eine Äquivalenzrelation.\footnote{Übungsblatt 10} Wir bezeichnen mit $\frac{a}{s}$ die Äquivalenzklasse von $(a,s)$ und mit $S^{-1}A$ die Menge der Äquivalenzklassen.

Man stattet $S^{-1}A$ mit einer Ringstruktur aus via
\begin{align*}
& \frac{a}{s} + \frac{b}{t} = \frac{at + bs}{st} \\
& \frac{a}{s} \frac{b}{t} = \frac{ab}{st}.
\end{align*}
Diese ist wohldefiniert.\footnote{Übungsblatt 10} Der Ring $S^{-1}A$ wird {\sf Lokalisierung von $A$ bezüglich $S$} genannt.

Es gibt einen kanonischen Ringhomomorphismus
\begin{align*}
\pi \colon A &\to S^{-1}A \\
\qquad a &\mapsto \frac{a}{1}.
\end{align*}
\end{definition}

\begin{remark}\label{remark: first-properties-of-localisation}
Die Lokalisierung $\pi \colon A \to S^{-1}A$ hat folgende Eigenschaften:
\begin{enumerate}
\item $s \in S$ $\Rightarrow$ $\pi(s)$ ist invertierbar in $S^{-1}A$;
\item $\pi(a)=0$ $\Rightarrow$ $as = 0$ für ein $s \in S$;
\item Jedes Element von $S^{-1}A$ ist von der Form $\frac{\pi(a)}{\pi(s)}$ mit $a \in A$ und $s \in S$.
\end{enumerate}
\end{remark}

\begin{example}
\
\begin{enumerate}
\item {\bf Quotientenkörper:} Sei $A$ ein Integritätsbereich und $S = A \setminus \{0\}$. Dann wird $S^{-1}A$ Quotientenkörper von $A$ genannt.

\item {\bf Lokalisierung an einem Primideal:} Sei $A$ ein Ring und $\mathfrak{p} \subset A$ ein Primideal. Dann ist $S = A \setminus \mathfrak{p}$ eine multiplikativ abgeschlossene Teilmenge. Die Lokalisierung $S^{-1}A$ bezeichnet man üblicherweise mit $A_{\mathfrak{p}}$.

\item {\bf Lokalisierung an einem Element:} Sei $A$ ein Ring und $f \in A$ ein Element. Dann ist $S = \{1, f, f^2, \dots \}$ eine multiplikativ abgeschlossene Teilmenge. Die Lokalisierung $S^{-1}A$ bezeichnet man üblicherweise mit $A_f$.

\item Was ist $\bC[t]_t$? Das ist der Ring von Laurent-Polynomen $\bC[t, t^{-1}]$.

\item Wenn gilt $S=\{ 1 \}$, dann bekommen wir $S^{-1}A = A$.

\item $S^{-1}A = 0 \quad \iff \quad 0 \in S$.

\end{enumerate}
\end{example}

\begin{proposition}
Sei $f\colon A \to B$ ein Ringhomomorphismus, sodass $f(S) \subset B^*$. Dann lässt sich $f$ eindeutig über die Lokalisierung $S^{-1}A$ faktorisieren
\begin{align*}
\xymatrix{
A \ar[r]^f \ar[d]^{\pi} & B \\
S^{-1}A \ar[ur]_{\exists ! \bar{f}}
}
\end{align*}
\end{proposition}

\begin{proof}
Aus der Kommutativität des Diagramms folgt sofort, dass für den Homomorphismus $\bar{f}$ gelten muss
\begin{align*}
\bar{f}\Big(\frac{a}{s}\Big) = f(a)f(s)^{-1}.
\end{align*}
Man muss nur überprüfen, dass durch diese Formel $\bar{f}$ wohldefiniert ist, d.h. aus $a/s = b/t$ folgt $\bar{f}(a/s)=\bar{f}(b/t)$. Wir haben
\begin{align*}
& a/s = b/t \iff \exists u \in S \colon (at-bs)u=0 \Rightarrow  \\
& (f(a)f(t)-f(b)f(s))f(u) = 0 \Rightarrow f(a)f(s)^{-1} = f(b)f(t)^{-1}.
\end{align*}
\end{proof}

Aus dem Satz ziehen wir jetzt ein Korollar, welches ähnlich zu Bemerkung~\ref{remark: first-properties-of-localisation} aussieht.
\begin{cor}
Sei $g \colon A \to B$ ein Ringhomomorphismus mit den Eigenschaften
\begin{enumerate}
\item $s \in S$ $\Rightarrow$ $g(s)$ ist invertierbar in $B$;
\item $g(a)=0$ $\Rightarrow$ $as = 0$ für ein $s \in S$;
\item Jedes Element von $B$ ist der Form $\frac{g(a)}{g(s)}$ mit $a \in A$ und $s \in S$.
\end{enumerate}
Dann gibt es einen eindeutigen Isomorphismus $h \colon S^{-1}A \to B$, sodass $g = h \circ \pi$.
\end{cor}

\begin{proof}
Übungsblatt 10.
\end{proof}

\subsection{Lokalisierung von Moduln}

\begin{definition}
Seien $A$ ein Ring, $S \subset A$ eine multiplikativ abgeschlossene Teilmenge, und $M$ ein $A$-Modul. Man definiert $S^{-1}M$ analog zu $S^{-1}A$. Erst definiert man eine Äquivalenzrelation auf $M \times S$
\begin{align*}
& (m,s) \sim (n,t) \iff (mt-ns)u = 0 \quad \text{für ein }  u \in S,
\end{align*}
dann die Gruppenstruktur
\begin{align*}
& \frac{m}{s} + \frac{n}{t} = \frac{mt + ns}{st},
\end{align*}
dann die $S^{-1}A$-Modul Struktur
\begin{align*}
& \frac{a}{s} \cdot \frac{m}{t} = \frac{am}{st}.
\end{align*}

Gegeben ein Homomorphismus von $A$-Moduln $f \colon M_1 \to M_2$, definiert man einen induzierten Morphismus auf Lokalisierungen als
\begin{align*}
& S^{-1}f \colon S^{-1}M_1 \to S^{-1}M_2 \\
& \qquad \qquad \frac{x}{s} \mapsto \frac{f(x)}{s}
\end{align*}

Es ist leicht zu sehen, dass wir auf diese Weise einen Funktor
\begin{align*}
S^{-1}  \colon A \mhyphen \Mod \to S^{-1}A \mhyphen \Mod
\end{align*}
bekommen.
\end{definition}

\begin{example}
Wie für Ringe gibt es zwei wichtige Klassen von Beispielen:
\begin{enumerate}
\item Für ein Primideal $\mathfrak{p} \subset A$ haben wir $M_{\mathfrak{p}}$;

\item Für ein Element $f \in A$ haben wir $M_f$.
\end{enumerate}
\end{example}


\begin{proposition}\label{satz-localization-is-exact}
Der Lokalisierungsfunktor ist exakt.
\end{proposition}

\begin{proof}
Betrachten wir die exakte Sequenz
\begin{align*}
M' \overset{f}{\to}  M \overset{g}{\to} M''.
\end{align*}
Aus $g \circ f = 0$ folgt $S^{-1} g \circ S^{-1} f = 0$. Somit haben wir $\Im S^{-1} f \subset \Ker S^{-1} g$.

Jetzt zeigen wir die Inklusion $\Ker S^{-1} g \subset \Im S^{-1} f$. Sei $m/s \in \Ker S^{-1} g$, d.h. $g(m)/s = 0$ in $S^{-1} M''$. Dann $\exists t \in S$, sodass gilt $t g(m)=0$ in $M''$. Daraus folgt $g(tm)=0$ und somit gilt $tm \in \Ker g = \Im f$. Nun gibt es ein $m' \in M'$, sodass $f(m')=tm$. Somit gilt Folgendes in $S^{-1} M$:
$$
\frac{m}{s} = \frac{f(m')}{st} = (S^{-1} f)\Big(\frac{m'}{st}\Big) \in \Im S^{-1} f.
$$
\end{proof}

\begin{proposition}\label{satz-localization-as-extension-of-scalars}
Es gibt einen Isomorphismus von $S^{-1} A$-Moduln
$$
S^{-1} A \otimes_A M \cong S^{-1} M
$$
gegeben durch
$$
\frac{a}{s} \otimes m \mapsto \frac{am}{s}.
$$
\end{proposition}

\begin{proof}
Die Surjektivität ist klar. Wir zeigen nur die Injektivität. Sei $x \in S^{-1} A \otimes_A M$ ein beliebiges Element
\begin{align*}
x = \sum_{i=1}^n \frac{a_i}{s_i} \otimes m_i.
\end{align*}
Man kann $x$ auf folgende Weise umschreiben
\begin{align*}
& x = \sum_{i=1}^n \frac{a_i}{s_i} \otimes m_i = \sum_{i=1}^n \frac{1}{s_i} \otimes \widetilde{m}_i =  \sum_{i=1}^n \frac{1}{s} \otimes \widetilde{\widetilde{m}_i} = \frac{1}{s} \otimes \sum \widetilde{\widetilde{m}_i} \\
& \widetilde{m}_i = a_i m_i \, , \quad  \widetilde{\widetilde{m}_i} = t_i \widetilde{m}_i \, , \quad  s = s_1 \dots s_n \, , \quad t_i = s_1 \dots s_{i-1} \hat{s_i} s_{i+1} \dots s_n.
\end{align*}
Somit reicht es, die Elemente der Form $x=\frac{1}{s} \otimes m$ zu betrachten. Das Bild von $\frac{1}{s} \otimes m$ ist $m/s$. Wir haben
\begin{align*}
\frac{m}{s} = 0 \iff \exists u \in S \colon mu=0 \Rightarrow \frac{1}{s} \otimes m = \frac{u}{su} \otimes m = \frac{1}{su} \otimes um = 0.
\end{align*}
\end{proof}

\begin{cor}
$S^{-1} A$ ist ein flacher $A$-Modul.
\end{cor}

\begin{proof}
Nach Satz \ref{satz-localization-as-extension-of-scalars} ist der Funktor $S^{-1}A\otimes_A \blank$ isomorph zum Lokalisierungsfunktor $S^{-1}(\blank)$. Letzterer ist exakt nach Satz \ref{satz-localization-is-exact}.
\end{proof}

\begin{proposition}\label{satz-distributivity-of-localization}
Seien $M, N$ zwei $A$-Moduln. Dann gibt es einen Isomorphismus von $S^{-1}A$-Moduln
$$S^{-1}M\otimes_{S^{-1}A}S^{-1}N\overset{\cong}{\longrightarrow}S^{-1}(M\otimes_A N)$$
gegeben durch (d.h.\ als lineare Fortsetzung von)
$$\frac{m}{s}\otimes \frac{n}{t} \longmapsto \frac{m\otimes n}{st}.$$
\end{proposition}
\begin{proof}
Wir wenden Satz \ref{satz-localization-as-extension-of-scalars} und die Eigenschaften des Tensorprodukts aus Satz \ref{satz-properties-of-tensor-product} an. (Man beachte, dass Assoziativität auch gilt, wenn die Tensorprodukte über verschiedenen Ringen gebildet werden, s.\ dazu [AM], Exercise 2.15 on p.~27)
\begin{align*}
S^{-1}M\otimes_{S^{-1}A} S^{-1}N &\cong (S^{-1}A\otimes_A M)\otimes_{S^{-1}A}(S^{-1}A\otimes_{A}N)\\
&\cong \left((S^{-1}A\otimes_A M)\otimes_{S^{-1}A}S^{-1}A\right)\otimes_A N\\
&\cong (S^{-1}A\otimes_A M)\otimes_A N\\
&\cong S^{-1}A\otimes_A (M\otimes_A N)\cong S^{-1}(M\otimes_A N).
\end{align*}
Da wir alle diese Isomorphismen explizit kennen, lässt sich leicht sehen, dass der Isomorphismus durch obige Vorschrift gegeben ist.
\end{proof}

\begin{proposition}
Sei $M$ ein $A$-Modul und seien $N,P\subset M$ Untermoduln. Dann gilt:
\begin{enumerate}
	\item $S^{-1}(N+P)=S^{-1}N+S^{-1}P$
	\item $S^{-1}(N\cap P)=S^{-1}N\cap S^{-1}P$
	\item $S^{-1}(M/N)\cong (S^{-1}M)/(S^{-1}N)$ als $S^{-1}A$-Moduln.
\end{enumerate}
\end{proposition}
\begin{proof}
Der erste Punkt ist klar nach Definition.\\
Zu Punkt 2: Die Inklusion $\subset$ ist klar. Sei also $w\in S^{-1}N\cap S^{-1}P$, d.h.\ wir können schreiben $w=\frac{n}{s}=\frac{p}{t}$ für gewisse $n\in N$, $p\in P$, $s,t\in S$. Somit ist $(nt-ps)u=0$ für ein $u\in S$. Es folgt $ntu=psu$ und dieses Element liegt daher in $N\cap P$. Wir sehen, dass gilt $w=\frac{ntu}{stu}$ und daher ist $w\in S^{-1}(N\cap P)$.\\
Zu Punkt 3: Wende den exakten Funktor $S^{-1}$ auf die kurze exakte Sequenz $0\to N\to M\to M/N\to 0$ an.
\end{proof}

\subsection{Lokale Eigenschaften}

Eine Eigenschaft eines Moduls (oder auch Homomorphismus) heißt \emph{lokal}, falls sich ihre Gültigkeit überprüfen lässt, indem man sie für alle Lokalisierungen an Primidealen testet. Die folgenden drei Sätze geben drei solche Eigenschaften an: Die Eigenschaft, der Nullmodul zu sein, ist lokal. Ebenso sind Injektivität und Flachheit lokale Eigenschaften.

\begin{proposition}\label{satz-being-zero-is-local}
Sei $M$ ein $A$-Modul. Die folgenden Aussagen sind äquivalent:
\begin{enumerate}
	\item $M=0$;
	\item $M_\mathfrak{p}=0$ für alle Primideale $\mathfrak{p}\subset A$;
	\item $M_\mathfrak{m}=0$ für jedes maximale Ideal $\mathfrak{m}\subset A$.
\end{enumerate}
\end{proposition}
\begin{proof}
Offensichtlich folgt 2.\ aus 1. Da maximale Ideale prim sind, folgt auch 3.\ aus 2. Es bleibt also zu zeigen 3.\ $\Rightarrow$ 1.:
Angenommen, $M\neq 0$, dann existiert ein $x\in M$ mit $x\neq 0$. Sei $\mathfrak{a}:=\mathrm{ann}(x):= \{r\in R\mid rx=0\}\subset R$ das Annihilatorideal von $x$. Es gilt $ \mathfrak{a}\neq R$ (denn sonst wäre $1\cdot x=0$). Also ist $\mathfrak{a}$ in einem maximalen Ideal $\mathfrak{m}$ enthalten (nach Korollar \ref{cor-ideal-contained-in-maximal}).
Nach Voraussetzung ist $M_\mathfrak{m}=0$, d.h.\ in $M_\mathfrak{m}$ gilt $\frac{x}{1}=\frac{0}{1}$. Dies bedeutet $(x\cdot 1 - 0\cdot 1)u=0$ (also $xu=0$) für ein $u\in R\setminus \mathfrak{m}$. Somit wäre $u$ ein Ringelement, welches $x$ annihiliert, aber nicht in $\mathfrak{m}$, also insbesondere nicht in $\mathfrak{a}$ liegt. Widerspruch, da $\mathfrak{a}$ das Annihilatorideal ist.
\end{proof}

\begin{proposition}\label{satz-injectivity-is-local}
Sei $\varphi\colon M\to N$ ein $A$-Modulhomomorphismus. Die folgenden Aussagen sind äquivalent:
\begin{enumerate}
	\item $\varphi$ ist injektiv;
	\item $\varphi_\mathfrak{p}$ ist injektiv für alle Primideale $\mathfrak{p}\subset A$;
	\item $\varphi_\mathfrak{m}$ ist injektiv für jedes maximale Ideal $\mathfrak{m}\subset A$.
\end{enumerate}
\end{proposition}
\begin{proof}
1.\ $\Rightarrow$ 2.: Sei $\varphi$ injektiv, dann ist die Sequenz $0\to M\overset{\varphi}{\to} N$ exakt, also (nach Satz \ref{satz-localization-is-exact}) auch die Sequenz $0\to M_\mathfrak{p}\overset{\varphi_\mathfrak{p}}{\to}N_\mathfrak{p}$ für ein beliebiges Primideal $\mathfrak{p}\subset A$. Also ist $\varphi_\mathfrak{p}$ injektiv.\\
2.\ folgt offensichtlich aus 3.\\
3.\ $\Rightarrow$ 1.: Betrachte die exakte Sequenz $0\to \Ker \varphi \to M \overset{\varphi}{\to}N$. Es ist damit auch $0\to (\Ker\varphi)_\mathfrak{m} \to M_\mathfrak{m}\overset{\varphi_\mathfrak{m}}{\to}N_\mathfrak{m}$ exakt und es folgt, dass $(\Ker \varphi)_\mathfrak{m}\cong \Ker(\varphi_\mathfrak{m})$ (für ein beliebiges maximales Ideal $\mathfrak{m}\subset A$). Nach Voraussetzung (3.) ist $\Ker(\varphi_\mathfrak{m})=0$, also folgt $\Ker\varphi=0$ nach Satz \ref{satz-being-zero-is-local}.
\end{proof}
Eine analoge Aussage gilt, wenn man \glqq injektiv\grqq\, jeweils an jeder Stelle ersetzt durch \glqq surjektiv\grqq, \glqq bijektiv\grqq\, oder \glqq die Nullabbildung\grqq.

\begin{proposition}
Sei $M$ ein $A$-Modul. Die folgenden Aussagen sind äquivalent:
\begin{enumerate}
	\item $M$ ist ein flacher $A$-Modul;
	\item $M_\mathfrak{p}$ ist ein flacher $A_\mathfrak{p}$-Modul für alle Primideale $\mathfrak{p}\subset A$;
	\item $M_\mathfrak{m}$ ist ein flacher $A_\mathfrak{m}$-Modul für alle maximalen Ideale $\mathfrak{m}\subset A$.
\end{enumerate}
\end{proposition}
\begin{proof}
1.\ $\Rightarrow$ 2.: Sei $M$ ein flacher $A$-Modul. Wir wollen zeigen, dass der Funktor $M_\mathfrak{p}\otimes_{A_\mathfrak{p}}\blank$ exakt ist. Setzen wir $S=A\setminus \mathfrak{p}$, so gilt:
\begin{align*}
M_\mathfrak{p}\otimes_{A_\mathfrak{p}}\blank\cong S^{-1}M\otimes_{S^{-1}A}\blank \cong (M\otimes_A S^{-1}A)\otimes_{S^{-1}A}\blank \cong M\otimes_A(S^{-1}A\otimes_{S^{-1}A}\blank)\cong M\otimes_A \blank,
\end{align*} also sind $M_\mathfrak{p}\otimes_{A_\mathfrak{p}}\blank$ und $M\otimes_A \blank$ isomorph als Funktoren auf $A_\mathfrak{p}$-Moduln (welche insbesondere $A$-Moduln sind, daher macht diese Aussage Sinn).
Von letzterem Funktor war die Exaktheit vorausgesetzt.\\
2.\ $\Rightarrow$ 3.: klar.\\
3.\ $\Rightarrow$ 1.: Zu zeigen ist: Der Funktor $M\otimes_A\blank$ ist exakt. Da Tensorprodukte immer rechtsexakt sind, reicht es zu zeigen, dass der Funktor Injektivität erhält. Sei also $N\overset{\varphi}{\to}P$ ein injektiver $A$-Modulhomomorphismus.\\ Dann ist $N_\mathfrak{m}\overset{\varphi_\mathfrak{m}}{\to}P_\mathfrak{m}$ injektiv für alle maximalen Ideale $\mathfrak{m}\subset A$ (da Lokalisieren exakt ist). Nach Voraussetzung (3.) ist $M_\mathfrak{m}\otimes_{A_\mathfrak{m}}N_\mathfrak{m}\overset{\id_{M_\mathfrak{m}}\otimes \varphi_\mathfrak{m}}{\longrightarrow}M_\mathfrak{m}\otimes_{A_\mathfrak{m}}P_\mathfrak{m}$ immer noch injektiv, somit auch $(M\otimes_A N)_\mathfrak{m}\overset{(\id_M\otimes \varphi)_\mathfrak{m}}{\longrightarrow}(M\otimes_A P)_\mathfrak{m}$ wegen Satz \ref{satz-distributivity-of-localization}. Schließlich folgt die Injektivität von $M\otimes_A N\to M\otimes_A P$ aus Satz \ref{satz-injectivity-is-local}.
\end{proof}


\subsection{Erweiterung und Kontraktion von Idealen}

Sei $A$ ein Ring, $S\subset A$ eine multiplikativ abgeschlossene Teilmenge und $i\colon A\to S^{-1}A, a\mapsto \frac{a}{1}$ die natürliche Abbildung.

\begin{definition}
Für ein Ideal $\mathfrak{a}\subset A$ heißt $S^{-1}\mathfrak{a}\subset S^{-1}A$ seine {\sf Erweiterung} in $S^{-1}A$.

Für ein Ideal $\mathfrak{b}\subset S^{-1}A$ heißt $i^{-1}\mathfrak{b}\subset A$ seine {\sf Kontraktion} in $A$.
\end{definition}

Man beachte hierbei, dass ein Ideal $\mathfrak{a}\subset A$ ein $A$-Modul ist (genauer ein $A$-Untermodul von $A$ selbst). Somit meint $S^{-1}\mathfrak{a}$ die Lokalisierung als Modul.

Es ist leicht zu sehen, dass Erweiterung und Kontraktion wieder Ideale sind.

\begin{proposition}
\begin{enumerate}
	\item Jedes Ideal $\mathfrak{b}\subset S^{-1}A$ ist von der Form $\mathfrak{b}=S^{-1}\mathfrak{a}$ für ein $\mathfrak{a}\subset A$.
	\item Es gibt eine 1:1-Korrespondenz (Bijektion von Mengen), gegeben durch Kontraktion und Erweiterung
	\begin{align*}
	\{\text{Primideale in $S^{-1}A$}\}&\overset{\text{1:1}}{\longleftrightarrow}\{\text{zu $S$ disjunkte Primideale in $A$}\}\\
	\mathfrak{q}&\longmapsto i^{-1}(\mathfrak{q})\\
	S^{-1}\mathfrak{p}&\longmapsfrom \mathfrak{p}
	\end{align*}
\end{enumerate}
\end{proposition}
\begin{proof}
Zu 1.: Setze $\mathfrak{a}:=i^{-1}(\mathfrak{b})$. Wir zeigen die Gleichheit $\mathfrak{b}=S^{-1}(\mathfrak{a})$. Sei also $b=\frac{x}{s}\in \mathfrak{b}$. Dann gilt (da $\mathfrak{b}$ ein Ideal ist) $\frac{x}{1}=s\cdot \frac{x}{s}\in \mathfrak{b}$ und damit $x\in \mathfrak{a}$. Also ist $\frac{x}{s}\in S^{-1}\mathfrak{a}$. Die andere Inklusion ist noch einfacher: Ist $\frac{x}{s}\in S^{-1}(i^{-1}(\mathfrak{b}))$ für ein ein $x\in i^{-1}(\mathfrak{b})$ (d.h.\ $\frac{x}{1}\in \mathfrak{b}$), dann ist auch $\frac{x}{s}=\frac{1}{s}\cdot \frac{x}{1}\in \mathfrak{b}$, da $\mathfrak{b}$ ein Ideal in $S^{-1}A$ ist.\\
Zu 2.: Wir zeigen zunächst die Wohldefiniertheit der beiden Abbildungen.
Die Abbildung von links nach rechts ist wohldefiniert, da die Kontraktion eines Primideals $\mathfrak{q}$ wieder ein Primideal ist (Satz \ref{satz-preimages-of-prime-ideals}). Außerdem kann dieses keine Elemente aus $S$ enthalten, sonst wäre $1 \in \mathfrak{q}$.\\
Die Abbildung von rechts nach links ist wohldefiniert, d.h.\ $S^{-1}\mathfrak{p}$ ist ein Primideal: Seien $\frac{a}{s},\frac{b}{t}\in S^{-1}A$ ($a,b\in A$, $s,t\in S$) mit $\frac{ab}{st}\in S^{-1}\mathfrak{p}$. Das bedeutet, dass $\frac{ab}{st}=\frac{p}{u}$ für ein $p\in \mathfrak{p}$ und ein $u\in S$, also gilt $(abu-pst)v=0$ in $A$ für ein $v\in S$. Es folgt somit $abuv=pstv$ und die rechte Seite liegt sicher in $\mathfrak{p}$. Da $\mathfrak{p}$ Primideal war, muss einer der Faktoren $a,b,u,v$ in $\mathfrak{p}$ liegen. Für $u$ und $v$ ist das nicht möglich, da diese in $S$ liegen und $\mathfrak{p}$ disjunkt zu $S$ vorausgesetzt war. Schließlich folgt $\frac{a}{s}\in S^{-1}\mathfrak{p}$ oder $\frac{b}{t}\in S^{-1}\mathfrak{p}$.\\
Die beiden Abbildungen sind invers zueinander: $S^{-1}(i^{-1}(\mathfrak{q}))$ gilt nach Teil 1. Außerdem ist $i^{-1}(S^{-1}\mathfrak{p})$ für ein zu $S$ disjunktes Primideal $\mathfrak{p}$ (Übungsblatt 11).
\end{proof}

Insbesondere bedeutet dies für die Lokalisierung an einem Primideal das Folgende.
\begin{cor}
Sei $\mathfrak{p}\subset A$ ein Primideal. Kontraktion und Erweiterung geben eine Bijektion
$$\{\text{Primideale in $A_\mathfrak{p}$}\}\overset{\text{1:1}}{\longleftrightarrow}\{\text{Primideale $I$ in $A$ mit $I\subset \mathfrak{p}$}\}$$
\end{cor}

\newpage

\section{Das Spektrum eines Rings}
\subsection{Zariski-Topologie}

Sei $R$ ein kommutativer Ring mit $1$. In Abschnitt \ref{subsection-Spektrum} haben wir bereits die Zariski-Topologie auf der Menge aller Primideale $\Spec R$ gesehen, deren abgeschlossene Mengen genau solche von der Form
$$V(I)=\{ \mathfrak{p}\in \Spec R\mid I\subset \mathfrak{p} \}$$
für ein beliebiges Ideal $I\subset R$ sind.

Die offenen Mengen der Zariski-Topologie sind folglich genau solche von der Form
$$D(I):=\Spec R\setminus V(I)=\{ \mathfrak{p}\in \Spec R\mid I\not\subset \mathfrak{p} \}.$$
Für diese gilt analog zu Lemma \ref{lemma-V(I)-define-topology}:
\begin{lemma}
\begin{align*}
& D(0) = \emptyset \quad \text{und}\quad D(R) = \Spec R \\
& \cup_s D(I_i)  = D (\sum_s I_s) \\
& D(I) \cap D(J) = D(IJ) \\
\end{align*}
\end{lemma}
Ist $f\in R$ und $(f)\subset R$ das von $f$ erzeugte Ideal, so schreiben wir $D(f):=D((f))=\{ \mathfrak{p}\in \Spec R\mid f\notin \mathfrak{p} \}$. Offene Mengen dieser Form haben die folgende schöne Eigenschaft:
\begin{proposition}
Die Mengen der Form $D(f)$ für ein $f\in R$ bilden eine {\sf Basis} der Zariski-Topologie, d.h.\ jede offene Menge lässt sich als Vereinigung von Mengen der Form $D(f)$ schreiben.
\end{proposition}
\begin{proof}
Der Beweis ist eine einfache Rechnung: Sei $I\subset R$ ein Ideal. Dann ist
$$D(I)=D(\sum_{f\in I}(f))=\cup_{f\in I}D(f).$$
\end{proof}

\begin{remark}
	Die Zariski-Topologie hat folgende Eigenschaften:
\begin{itemize}
	\item Die einpunktige Menge $\{ \mathfrak{p} \}$ ist genau dann abgeschlossen, wenn $\mathfrak{p}$ ein maximales Ideal ist.
	\item Ist $R$ nullteilerfrei, so ist $(0)$ ein Primideal und es gilt: Der Abschluss der einpunktigen Menge $\{(0)\}$ ist ganz $\Spec R$.
	\item Die Zariski-Topologie ist im Allgemeinen nicht Hausdorff (d.h.\ für zwei verschiedene Punkte gibt es keine offenen Umgebungen, welche disjunkt sind).
\end{itemize}
\end{remark}

\subsection{Garben}

Sei $X$ ein topologischer Raum. Wir führen in diesem Kapitel den Begriff einer Garbe ein, mit dessen Hilfe wir lokale Informationen (z.B.\ lokale Funktionen auf einem Raum) besser handhaben können.

\begin{definition}
Sei $X$ ein topologischer Raum. Die {\sf Kategorie der offenen Mengen} $\mathbf{Op}_X$ ist gegeben durch die Objekte
$$Obj(\mathbf{Op}_X):=\{ U\subset X\mid U \text{ offen} \}$$
und für je zwei Objekte $U,V$ der Morphismenmenge
$$\mathrm{Hom}_{\mathbf{Op}_X}(U,V):=\begin{cases}
\{U\hookrightarrow V\}& \text{falls $U\subset V$}\\
\emptyset &\text{sonst}
\end{cases}$$
\end{definition}

\begin{definition}
Eine {\sf Prägarbe} $\mathcal{F}$ auf $X$ (mit Werten in einer Kategorie $\cC$) ist ein kontravarianter Funktor von $\mathbf{Op}_X$ nach $\cC$.
\end{definition}
Meist wird $\cC$ eine Kategorie von Mengen mit Struktur (abelsche Gruppen, Ringe, $R$-Moduln etc.) sein. Ist z.B.\ $\cC=\mathit{Ab}$, so besteht eine Prägarbe $\mathcal{F}$ auf $X$ mit Werten in $\cC$ aus:
\begin{itemize}
	\item einer abelschen Gruppe $\mathcal{F}(U)$ für jede offene Teilmenge $U\subset X$,
	\item einem Gruppenhomomorphismus $r_{UV}=\mathcal{F}(U\hookrightarrow V)\colon \mathcal{F}(V)\to \mathcal{F}(U)$
	für jede Teilmengenbeziehung $U\subset V$,
\end{itemize}
sodass gilt (Funktoraxiome):
\begin{itemize}
	\item $r_{UU}=\id_{\mathcal{F}(U)}$,
	\item $r_{UV}\circ r_{VW}=r_{UW}$ für jede Teilmengenbeziehung $U\subset V\subset W$.
\end{itemize}
Elemente von $\mathcal{F}(U)$ heißen {\sf Schnitte} von $\mathcal{F}$ auf $U$ und $r_{UV}$ wird oft als {\sf Restriktionsabbildung} bezeichnet.

\begin{example}\label{beispiel-presheaf}\text{}\vspace*{-0.8cm}\\
\begin{enumerate}
	\item konstante Prägarbe:\\ $\mathcal{F}(U):=\mathbb{Z}$ für alle $U\subset X$ offen.\\ Restriktionsabbildungen sind Identitäten. (Anstelle von $\mathbb{Z}$ kann ein beliebiges Objekt verwendet werden.)
	\item Prägarbe der stetigen reellwertigen Funktionen:\\ $\mathcal{F}(U):=\{ f\colon U\to \mathbb{R} \mid f \text{ ist stetig} \}$\\
	und für $U\subset V$ ist $r_{UV}\colon \mathcal{F}(V)\to \mathcal{F}(U)$ gegeben durch $f\mapsto f|_U$.
	\item Prägarbe der beschränkten Funktionen (analog zu 2.)
\end{enumerate}
\end{example}
Erfüllt eine Prägarbe ein zusätzliches Axiom, so bezeichnen wir sie als Garbe. Wir formulieren dieses Axiom nun in der Situation, dass $\cC$ die Kategorie abelscher Gruppen (oder Ringen) ist. In diesem Fall können wir von \emph{Elementen} von $\mathcal{F}(U)$ sprechen und die Nullgruppe (oder der Nullring), das terminale Objekt von $\cC$ ist.
\footnote{For more general categories see \url{https://en.wikipedia.org/wiki/Sheaf_(mathematics)}}

\begin{definition}\label{def-sheaf}
Eine Prägarbe $\mathcal{F}$ (auf $X$ mit Werten in $\cC$) heißt {\sf Garbe}, falls gilt:\\
Ist $U\subset X$ eine offene Teilmenge und $U=\bigcup_{i\in I}U_i$ eine beliebige offene Überdeckung von $U$ und ist weiter $s_i\in \mathcal{F}(U_i)$ gegeben für alle $i\in I$, sodass die $s_i$ eine \emph{kompatible Familie} bilden, d.h.\ für alle $i,j\in I$ gelte $r_{U_i\cap U_j,U_i}(s_i)=r_{U_i\cap U_j,U_j}(s_j)$,
dann existiert ein eindeutiges $s\in \mathcal{F}(U)$, sodass $r_{U_i,U}(s)=s_i$.

\smallskip

Insbesondere für die leere Teilmenge gilt es immer $\mathcal{F}(\emptyset) = 0$.
\end{definition}
Die Idee dieser Eigenschaft ist also: \glqq Aus lokalen Daten lässt sich das globale Datum (eindeutig) rekonstruieren\grqq. Man sagt auch oft: Lokale Schnitte (welche kompatibel sind, also jeweils auf der Überlappung $U_i\cap U_j$ übereinstimmen) lassen sich eindeutig zum einem globalen Schnitt \glqq verkleben\grqq.

\begin{example}
\
\begin{enumerate}
	\item Die konstante Prägarbe aus Beispiel \ref{beispiel-presheaf} ist keine Garbe: $F(\emptyset) \neq 0$.
	
	\smallskip
	
	Sogar wenn wir die Definition der konstanten Prägarbe anpassen würden, sodass gilt $F(\emptyset) = 0$.
	Dann können wir folgendes Argument anwenden.
	 Betrachte dazu eine unzusammenhängende Menge $U=U_1\cup U_2$ (mit $U_1\cap U_2=\emptyset$). Ein lokaler Schnitt $s_1\in\mathcal{F}(U_1)$ ist eine ganze Zahl, ebenso $s_2\in \mathcal{F}(U_2)$ (Kompatibilität ist hier eine leere Bedingung, also immer erfüllt). Sind aber $s_1$ und $s_2$ verschieden, so gibt es kein passendes $s\in \mathcal{F}(U)$ (denn dies müsste auch eine (einzelne) ganze Zahl sein).
	 
 	\smallskip
	 
	 Im Buch von Ravi Vakil \cite{Va} im Abschnitt 2.2.10 gibt es eine interessant Diskussion zu diesem Beispiel.
	 In Abschnitten 2.2.1--2.2.10 finden Sie mehr zu Garben.

	Es gibt allerdings auch die sogenannte \emph{konstante Garbe}: Sie kann zum Beispiel definiert werden als die (Prä-)Garbe der lokal konstanten Funktionen. Dann wäre in obigem Beispiel $\mathcal{F}(U)=\mathbb{Z}^2$, die Restriktionsabbildungen zu $\mathcal{F}(U_1)$ und $\mathcal{F}(U_2)$ wären die Projektionen auf die erste bzw.\ zweite Komponente und wir könnten somit $s:=(s_1,s_2)$ wählen.
	\item Die Prägarbe der stetigen Funktionen ist eine Garbe.
	\item Die Prägarbe der beschränkten Funktionen erfüllt das Garbenaxiom nicht: Sei z.B.\ $X=\mathbb{R}$, wähle $U=\mathbb{R}$, die Überdeckung $U=\bigcup_{k\in \mathbb{Z}} U_k$, wobei $U_k$ das Intervall $(k,k+2)$ bezeichnet, und sei $s_k\in \mathcal{F}(U_k)$ gegeben durch die Funktion $f(x)=x^2$. Dann würden die $s_k$ sich zwar eindeutig zu einer Funktion $U\to \mathbb{R}$ verkleben lassen (nämlich zur Funktion $x^2$), diese ist aber nicht beschränkt auf ganz $U$ und somit existiert kein passendes $s\in \mathcal{F}(U)$.
\end{enumerate}
\end{example}
Grundsätzlich gilt: Prägarben von Funktionen mit bestimmten Eigenschaften bilden genau dann eine Garbe, wenn die entsprechende Eigenschaft \emph{lokal} ist (d.h.\ lokal überprüft werden kann). Das ist beispielsweise bei Stetigkeit der Fall (\glqq Eine Funktion ist genau dann stetig, wenn sie an jedem Punkt stetig ist.\grqq), aber nicht bei Beschränktheit (eine lokal beschränkte Funktion muss nicht global beschränkt sein).

\begin{remark}
Das Garbenaxiom aus Definition \ref{def-sheaf} lässt sich auch anders formulieren:
Für jede offene Überdeckung $U=\bigcup_{i\in I}U_i$ ist die Sequenz
$$0\longrightarrow \mathcal{F}(U)\longrightarrow \prod_{i \in I}\mathcal{F}(U_i)\longrightarrow \prod_{i,j \in I} \mathcal{F}(U_i\cap U_j)$$
exakt.
Die Abbildungen sind hierbei gegeben durch $s\mapsto (r_{U_i,U}(s))_{i\in I}$ und $(s_i)_{i\in I}\mapsto (r_{U_i\cap U_j,U_i}(s_i)-r_{U_i\cap U_j,U_j}(s_j))$.\\ Ist $\cC$ keine Kategorie von Mengen mit Struktur (d.h.\ können wir nicht über Elemente sprechen), aber eine Kategorie, in der Produkte existieren, so kann man diese Abbildungen ebenfalls definieren (mithilfe der universellen Eigenschaft von Produkten).
\end{remark}

Unser Ziel ist es nun, $\Spec R$ mit einer Garbe $\mathcal{O}_{\Spec R}$ von \glqq guten\grqq\, Funktionen auszustatten. Betrachten wir das Beispiel $R=k[x]$ für einen algebraisch abgeschlossenen Körper $k$. Dann ist $\Spec R=\{ (x-a)\mid a\in k \}\cup \{ (0) \}$, d.h.\ abgesehen vom sogenannten generischen Punkt $(0)$ besteht das Spektrum aus einen Punkt für jedes Element von $k$. Funktionen auf ganz $\Spec R$ sollen Polynomfunktionen sein, d.h. $\mathcal{O}_{\Spec R}(\Spec R):=R$.\\ Jetzt fragen wir uns, was eine sinnvolle Definition für Funktionen auf kleineren offenen Mengen wäre, beispielsweise auf $D(f)$, $f\in k[x]$. $D(f)$ ist die Menge aller Primideale, die $f$ nicht enthalten. Abgesehen vom Ideal $(0)$ sind das alle $(x-a)$, wo $a$ keine Nullstelle von $f$ ist (beachte. dass $k$ algebraisch abgeschlossen ist und $f$ somit in Linearfaktoren zerfällt). Anschaulich hat $f$ also auf $D(f)$ keine Nullstelle. Eine sinnvolle Definition könnte also die folgende sein: Lokale Funktionen auf $D(f)$ sind gebrochen-rationale Funktionen, bei denen eine Potenz von $f$ im Nenner steht. Dies entspricht genau der Lokalisierung von $R$ an $f$ also $\mathcal{O}_{\Spec R}(D(f)):=R_f$.


\subsection{Strukturgarbe auf $\Spec A$}

Let us now define the structure sheaf of $\cO$ on $\Spec A$.
We define the sections of $\cO$ over an open subset $U \subset \Spec A$ as
\begin{equation*}
  s \colon U \to \sqcup_{\fp \in U} A_{\fp},
\end{equation*}
such that
\begin{enumerate}
  \item $s(\fp) \in A_{\fp}$ for all $\fp \in U$;
  \item locally it is given by a fraction of elements in $A$, i.e. for any $\fp \in U$
  there exists a neighborhood $V \subset U$ of $\fp$ and elements $a,f \in A$
  such that $s(\fq) = \frac{a}{f}$ and $f \notin \fq$ for all $\fq \in V$.
\end{enumerate}

Using pointwise addition, multiplication and unit, we see that the set $\cO(U)$
is a commutaive ring. We also have the natural restrictions $\cO(U) \to \cO(V)$
for $V \subset U$, which obviously satisfy the sheaf condition. Thus, we have
defined the structure sheaf
\begin{equation*}
  \cO \quad \text{or} \quad \cO_{\Spec A}.
\end{equation*}


\begin{definition}
  The pair $(\Spec A, \cO_{\Spec A})$ is called the \emph{spectrum of $A$.}
\end{definition}


\begin{proposition}\label{proposition:spec-A-structure-sheaf}
  Let $A$ be a ring and $(\Spec A, \cO)$ its spectrum. We have
  \begin{enumerate}
    \item For any $\fp \in \Spec A$ we have $\cO_{\fp} \cong A_{\fp}$.

    \item For any $f \in A$ we have $\cO(D(f)) \cong A_f$.

    \item In particular, we have $\Gamma(\Spec A, \cO) \cong A$.
  \end{enumerate}
\end{proposition}

\begin{proof}
  Maybe later. See \cite[Chapter II, Proposition 2.2]{Ha}.
\end{proof}

Above we used $\cO_{\fp}$, which stands for the stalk of $\cO$ at $\fp$.
See \cite[p.~62]{Ha} for details.


\newpage
\bibliographystyle{plain}
\bibliography{refs}


\end{document}
